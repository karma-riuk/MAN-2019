%{{{-----------------------------------Basics----------------------------------%

%document class definition
\documentclass[
    11pt,
    a4paper,
    oneside,
    headinlcude, footinclude,
    twoside,
]{report}


%essential packages
\renewcommand*\rmdefault{ppl}
\usepackage[top=2.5cm,bottom=2.5cm,left=3cm,right=3cm]{geometry}
\usepackage[english]{babel} \usepackage[T1]{fontenc} \usepackage[utf8]{inputenc} \usepackage{xcolor} \usepackage{amssymb}


%additional packages
\usepackage{amssymb}
\usepackage{amsmath}
\usepackage[framemethod=Tikz]{mdframed}
\usepackage{tikz}
\usepackage{enumerate}
\usepackage{pgf,tikz,pgfplots} % for the transfer from geogebra to tikz
\usepackage{mathrsfs}% for the transfer from geogebra to tikz
\usepackage{graphicx}
\usepackage[makeroom]{cancel}
\usepackage{fancyhdr}
\usepackage{makecell} %to have thick hlines in the front page
\usepackage{tabularx}

%}}}

%{{{-----------------------------------Macros----------------------------------%

\newcommand{\myImplies}[0]{\rightarrow}

\newcommand{\powerset}[1]{\mathcal{P}(#1)}

\newcommand{\tvect}[3]{%
   \ensuremath{\Bigl(\begin{smallmatrix}#1\#2\#3\end{smallmatrix}\Bigr)}}

\newcommand{\myVector}[3]{\begin{pmatrix}#1\#2\#3\end{pmatrix}}

\newcommand{\tq}[0]{\ \textrm{ t.q. }\ }

\newcommand{\markDate}[1]{\begin{flushright}#1\end{flushright}}

\newcommand{\cqfd}[0]{\begin{flushright}$\Box$\end{flushright}}

\renewcommand{\vec}[1]{\overrightarrow{#1}}

\renewcommand{\to}[0]{\longrightarrow}

\renewcommand{\bar}[1]{\overline{#1}}

\def\getangle(#1)(#2)#3{
    \begingroup
        \pgftransformreset
        \pgfmathanglebetweenpoints{\pgfpointanchor{#1}{center}}{\pgfpointanchor{#2}{center}}
        \expandafter\xdef\csname angle#3\endcsname{\pgfmathresult}
    \endgroup
}

\newcommand\Warning{
    \makebox[1.4em][c]{
    \makebox[-5.5pt][c]{\raisebox{.2em}{!}}
    \makebox[0pt][c]{\color{red}\huge$\bigtriangleup$}}
}
%}}}

%{{{----------------------------------Settings---------------------------------%

\title{Maths 1A - AICC}

\author{Arnò Fauconnet}

\setlength{\parindent}{0pt} %disable initial indent on first paragraph of sections in the whole doc

% the style of the boxes
\newmdenv[
        roundcorner=10pt,
        middlelinecolor=red,
        backgroundcolor=gray!15,
        linewidth=2pt,
        frametitlerule=true]{highlightBox}

% increases the space between paragraphs
\setlength{\parskip}{.3em}

\usetikzlibrary{calc}
\usetikzlibrary{arrows}%for the transfer from geogebra to tikz
\pgfplotsset{compat=1.15}


% Geogebras wierd colors xD
\definecolor{ffffff}{rgb}{1.,1.,1.}
\definecolor{qqqqff}{rgb}{0.,0.,1.}
\definecolor{qqffqq}{rgb}{0.,1.,0.}
\definecolor{xfqqff}{rgb}{0.4980392156862745,0.,1.}
\definecolor{ffqqtt}{rgb}{1.,0.,0.2}
\definecolor{ududff}{rgb}{0.30196078431372547,0.30196078431372547,1.} 
\definecolor{ffqqqq}{rgb}{1.,0.,0.} 
\definecolor{xdxdff}{rgb}{0.49019607843137253,0.49019607843137253,1.}
\definecolor{zzttqq}{rgb}{0.6,0.2,0.}
\definecolor{uuuuuu}{rgb}{0.26666666666666666,0.26666666666666666,0.26666666666666666}
\definecolor{wwccff}{rgb}{0.4,0.8,1.}
\definecolor{qqttcc}{rgb}{0.,0.2,0.8}
\definecolor{ffwwzz}{rgb}{1.,0.4,0.6}
\definecolor{ttqqqq}{rgb}{0.2,0.,0.}

\tikzstyle{every node}=[font=\large]

\graphicspath{ {Maths_1A/figures/} }

% header and footer settings
\pagestyle{fancy}
\fancyhf{}
\fancyhead[LE,LO]{Arnaud Fauconnet}
\fancyhead[CE,CO]{\textsc{Maths 1A}}
\fancyhead[RE,RO]{MAN - Printemps 2019}
\fancyfoot[CE,CO]{\leftmark}
\fancyfoot[LE,RO]{\thepage}

\renewcommand{\headrulewidth}{2pt}
\renewcommand{\footrulewidth}{1pt}
%}}}

%%%%%%%%%%%%%%%%%%%%%%%%%%%%%%%%%%%%%%%%%%%%%%%%%%%%%%%%%%%%%%%%%%%%%%%%%%%%%%
%----------------------------------------------------------------------------%
%-------------------------------Text starts here-----------------------------%
%----------------------------------------------------------------------------%
%%%%%%%%%%%%%%%%%%%%%%%%%%%%%%%%%%%%%%%%%%%%%%%%%%%%%%%%%%%%%%%%%%%%%%%%%%%%%%


\begin{document}

\begin{titlepage}
   \begin{center}
       \vspace*{\fill}

       {\Huge EPFL}\\ 
%----------------------------------------------------------------------------%
       \vfill
       {\huge MAN}\\ [1em]
       {\Large Mise à niveau}\\
%----------------------------------------------------------------------------%
        \vfill
        \begin{tabularx}{\textwidth}{X}
            \Xhline{3\arrayrulewidth}\\
        \end{tabularx}\\ [2em]
        {\Huge Maths 1A} \\ [1em]
        \textsc{\huge Prepa-031(a)} \\ [2em]
       \begin{tabularx}{\textwidth}{X}
            \Xhline{3\arrayrulewidth}\\
        \end{tabularx}
%----------------------------------------------------------------------------%
        \vspace{.7cm}
        {\large
        \begin{tabularx}{.9\textwidth}{Xr}
            \textit{Student:} & \textit{Professor:}\\
            Arnaud \textsc{Fauconnet} & Guido \textsc{Burmeister}
        \end{tabularx}}
%----------------------------------------------------------------------------%
        \vfill
        {\Large Printemps - 2019}

%----------------------------------------------------------------------------%
        \vfill
        \includegraphics[width=7cm]{epfl-logo}

       \vfill
   \end{center} 
\end{titlepage} 
\setcounter{chapter}{2}
\chapter{Polynôme réels}
\label{cha:polynome_reels}

\section{Définition et opérations}

\paragraph{Définition:}

Un polynôme en $x$ à coefficients réels est une combinaison linéaire de
puissance de $x$.
$$P(x) = a_{n}x^{n}+a_{n-1}x^{n-1}+...+a_{1}x^{1} + a_{0}$$
avec $a_{k} \in \mathbb{R}, k= 0, ..., n$

On note $P \in \mathbb{R}[\ x\ ]$ {\color{red} "ensemble de polynôme en $x$ à
coefficients réels"}

Le degré de $P$, noté $\deg P$, est la plus grande puissance de $x$ dont le
coefficient est non nul.

\paragraph{Convention}
\label{par:convention}

Le polynôme nul est de degré $-\infty$ 

Le sous-ensemble de $\mathbb{R}[\ x\ ]$ des polynômes de degré inférieur ou
égale à $n \in \mathbb{N}$ est notée $\mathbb{P}_{n} [\ x\ ]$

$$\mathbb{P}_{n}[\ x\ ] = \{ P \in \mathbb{R}[\ x\ ] | \deg P \leq n\}$$

\paragraph{Définition:}

La somme de 2 polynômes $P$ et $Q$ se note $P + Q$. On l'obtient en
additionnant  les coefficients d'une même puissance 

\paragraph{Exemple:}

\[
    \begin{split}
        P(x) & = x^{3} + 3 x^{2} + 5x - 6\\
        Q(x) &= -x^{3} - 2x^{2} + 3
    \end{split}
\]

Alors 
$$(P+Q) (x) = P(x) + Q(x) = x^{2} + 5x -3$$

\paragraph{Remarque:}

$$\deg (P+Q) \leq \max(\deg P, \deg Q)$$

\paragraph{Définition:}

L'amplification par $\lambda \in \mathbb{R}$ d'un polynôme $P$ donne un
polynôme noté $\lambda P$. On obtient en multipliant chaque coefficient par $\lambda$.

\paragraph{Exemple:}

$$P(x) = 3x^{2}+5x-6\quad \quad \lambda = -\frac{2}{3}$$

Alors $$(\lambda P)(x) = \lambda \cdot P(x) = -2x^{2} - \frac{10}{3}x + 4$$

\paragraph{Remarques:}

\begin{enumerate}

    \item 
        \[
            \begin{split}
                \text{Si } \lambda \neq 0 & \iff \deg(\lambda P) = \deg P\\
                \text{Si } \lambda = 0 & \iff \deg(\lambda P) = 0
            \end{split}
        \]

    \item Avec les lois (addition et amplification), $\mathbb{R}[\ x\ ]$ et
        $\mathbb{P}_{n}[\ x\ ]$ sont des espaces vectoriels.
\end{enumerate}



\paragraph{Définition:}

Multiplication par un monôme.

Soit $$P(x) = a_{n}x^{n} + a_{n-1}x^{n-1} + ... + a_{1}x + a_{0}$$ et $$Q(x) =
x^{n}$$
un monôme.

Leur produit est un polynôme. On l'obtient en distribuant la multiplication par
$x^{n}$.
$$x^{m}(a_{n}x^{n} + ...+ a_{0}) = a_{n}x^{n+m} + a_{n-1}x^{n-1+m} + ... +
a_{0}x^{m}$$

\paragraph{Remarque:}

$$\deg (x^{m}P) = m + \deg P$$


\paragraph{Définition:}

Soient $P$ et $Q$ deux polynômes. Leur produit noté $P \cdot Q$. On l'obtient
par distribution des produits et un regroupant les coefficients d'une même
puissance de $x$.

\paragraph{Exemple:}

$$P(x) = 3x^{2} + 5x -6 \quad Q(x) = -x^{3} -2x^{2} +3$$

Alors
\[
    \begin{split}
        (PQ)(x) & = P(x) \cdot Q(x)\\
        &= (3x^{2} + 5x - 6) \cdot(-x^{3} -2x^{2} + 3)\\
        &=-3x^{5}-6x^{4}+9x^{2} - 5x^{4} -10x^{3}+15x+6x^{3}+12x^{2}-18\\
        &=-3x^{5} -11 x^{4} -4x^{3} +21 x^{2} + 15x -18\\
    \end{split}
\]

\paragraph{Remarque:}

$$\deg (PQ) = \deg P + \deg Q$$

\paragraph{Définition:}

Soit $$P(x) = a_{n}x^{n}+ a_{n-1}\cdot x^{n-1}+ ... +  a_{0} \in \mathbb{R}[\
x\ ]$$

Alors
\[
    \begin{split}
        P: \mathbb{R} & \to \mathbb{R}\\
        x&\mapsto P(x) \quad \quad \text{{\color{red} "image par $P$ de x"}}
    \end{split}
\]


est une fonction polynomiale.

En particulier, l'évaluation de $P$ en $x_{0} \in \mathbb{R}$ s'écrit 
$$P(x_{0}) = a_{n}x_{0}^{n}+ a_{n-1}\cdot x_{0}^{n-1}+ ... +  a_{0} $$
{\color{red} $P$ évalué en $x_{0}$}



\paragraph{Exemple:}

$$P(x) = x^{2} -5x +6 \quad \text{ et } \quad x_{0} = -2$$

Alors $$P(x_{0}) = (-2)^{2} - 5 (-2) + 6 = 20$$


\section{Binôme de Newton}
\label{sec:binome_de_newton}

\paragraph{Définition:}

Le polynôme en $x$.
$$P_{n}(x) = (x+a)^{n}, \quad a \in \mathbb{R},  n \in \mathbb{R}$$
est appelé binôme de Newton ( $x+a :$ binôme)


Calculons...

 TODO: this table is bugged
\begin{highlightBox}[frametitle={Triangle de Pascal}]
$$ \begin{tabular}{rllll|llll}
n = 0: \quad& (x+a)^{0} = 1 &&&&1 &  &  & \\
n = 1: \quad& (x+a)^{1} = x &+ a&&&1 & 1 &  & \\
n = 2: \quad& (x+a)^{2} = x^{2} &+ 2ax &+ a^{2}&&1 & 2 & 1 & \\
n = 3: \quad& (x+a)^{3} = x^{3} &+ 3 ax^{2} &+ 3a^{2}x &+ a^{3}&1 & 3 & 3 & 1\\
\end{tabular} $$
\end{highlightBox}


\paragraph{Définition:}

On note $C^{k}_{n}$ le nombre de manières de choisir un sous-ensemble à $k$
éléments dans un ensemble à $n$ éléments.

On peut montre que $$C^{k}_{n} = \frac{n!}{k!\cdot (n-k)!}$$
où $$k!  k \cdot (k-1) \cdot (k-2) \cdot ... \cdot 2 \cdot 1$$
est dit "$k$-factorielle".

On pose  $$0! = 1$$ et $$C^{0}_{n} = 1 = C^{0}_{0}$$


\paragraph{Remarque:}
Une factorielle est vite très grande...

On calcul plutôt: 

\[
    \begin{split}
        C^{k}_{n} &= \frac{n!}{k! \cdot (n-k)!} = \frac{n \cdot (n-1) \cdot ... \cdot
        (n-k+1) \cdot {\color{red}\cancel{{\color{black} (n-k)!}}}}{k! {\color{red}\cancel{{\color{black} (n-k)!}}}}\\
        & = \frac{n \cdot (n-1) \cdot ... \cdot (n-k+1)}{k \cdot (k-1) \cdot ... \cdot 1}
        {\color{white} \quad \frac{{\color{red} \leftarrow k\text{ facteurs}}}{{\color{red} \leftarrow k\text{ facteurs}}}}
    \end{split}
\]

\paragraph{Propriétés:}

\begin{enumerate}
    \item $C^{k}_{n} = C^{n-k}_{n}, \quad k = 0, ..., n$
    \item $C^{k}_{n} = C^{k-1}_{n-1} + C^{k}_{n-1}, \quad k = 1, ..., n-1$
    \item $C^{0}_{n} + C^{1}_{n} + C^{2}_{n} +...+ C^{n-1}_{n} + C^{n}_{n} =
        \sum^{n}_{k=0} C^{k}_{n} = 2^{n} $

\end{enumerate}
\paragraph{Corollaire:}
    
Le développement du binôme de Newton donne:
$$(x+a)^{n} = C^{0}_{n} a^{0} x^{n} + C^{1}_{n} a^{1} x^{n-1} + C^{2}_{n} a^{2} x^{n-2}+ ...+ C^{k}_{n} a^{k} x^{n-k} +...+ C^{n-1}_{n} a^{n-1} x^{1} + C^{n}_{n} a^{n} x^{0} $$
$$= \sum^{n}_{k=0} C^{k}_{n} a^{k}x^{n-k} \quad \text{{\color{red}
remarque: il y a n+1 termes}}$$

En effet , dnas le développement de $$(x+a)^{n} = \underbrace{(x+a) \cdot (x+a) \cdot ...\cdot (x+a)}_{n \text{ facteurs } }$$
le terme $a^{k}x^{n-k}$ apparait $C^{k}_{n}$ fois, on a à choisir $k$ fois le a
et du coup on a $n-k$ fois le $x$.

\paragraph{Exemple:}

Développer
\[
    \begin{split}
        (x - 1)^{6} &= C^{0}_{6}(-1)^{0}x^{6} + C^{1}_{6}(-1)^{1}x^{5} + C^{2}_{6}(-1)^{2}x^{4} + C^{3}_{6}(-1)^{3}x^{3} + C^{4}_{6}(-1)^{4}x^{2} + C^{5}_{6}(-1)^{5}x^{1} + C^{6}_{6}(-1)^{6}x^{0}\\
        &= x^{6} - 6x^{5} + 15 x^{4} -20x^{3}+15x^{2} -6x +1 
    \end{split}
\]

\paragraph{Exemples:}

\begin{enumerate}
    \item Donner le coefficient de $x^{127}$ dans $(x+2)^{129}$.

        Le terme en $x^{127}$ est ($k = 2$)

        $$C^{2}_{129} 2^{2}x^{127} = \frac{129\cdot 128}{2-1} \cdot 2^{2} x^{127} =
        33024 x^{127}$$

    \item Terme en $x^{8}$ dans $(\overbrace{4x^{3}}^{x} + \overbrace{\frac{3}{x^{2}}}^{a})^{11}$
    
        Le terme général ($(k+1)^{e}$ terme) est $$C^{k}_{n} \left(\frac{3}{x^{2}}\right)^{k}
        \cdot (4x^{3})^{n-k} = C_{11}^{k} 3^{k} 4^{11-k} x^{-2k} x^{3(11-k)}$$

        Il faut trouver $$k \tq 33-5k = 8 \quad (k = 0, 1, ..., 11)$$
        $$k = 5$$

        D'où le terme en $x^{8} : C^{5}_{11}3^{5}4^{11-5}x^{8}= ...$
\end{enumerate}

\section{Zéro, schéma de Hörmer, multiplication}
\label{sec:zero_schema_de_hormer_multiplication}

\paragraph{Théorème:}

Soient $P$ un polynôme avec $\deg P \geq 1$ et $x_{0} \in \mathbb{R}$. Alors il
existe un unique polynôme $$F \tq P(x) = F(x) \cdot (x-x_{0}) + P(x_{0})$$

\paragraph{Remarques:}

\begin{itemize}
    \item $\deg F = \deg P -1$
    \item Il est un cas particulier de division euclidienne 
\end{itemize}


En effet, notons $$P(x) = a_{n}x^{n} + a_{n-1}x^{n-1} + ... + a_{1} x + a_{0}$$
et $$F(x) = b_{n-1}x^{n-1} +b_{n-2}x^{n-2} + ... + b_{0}$$

Alors 

\[
    \begin{split}
        P(x) &= F(x) \cdot (x-x_{0}) + r \quad \quad \quad \quad \quad \quad
        \quad \quad \quad \quad \quad \quad \quad \quad r\text{: à déterminer}\\
        &= b_{n-1}x^{n} + b_{n-2}x^{n-1} + ... + b_{0}x - x_{0}b_{n-1}x^{n-1} -
        x_{0}b_{n-2}x^{n-2} - ... -x_{0}b_{0}
    \end{split}
\]


 TODO: this table is bugged
$$
\begin{tabular}{rll}
b_{n-1} &= a_{n} & {\color{red} \cdot x^{n}_{0}}\\
b_{n-2} &= a_{n-1} + x_{0}b_{n-1} & {\color{red} \cdot x^{n-1}_{0}}\\
b_{n-3} &= a_{n-2} + x_{0}b_{n-2} & {\color{red} \cdot x^{n-2}_{0}}\\
&\vdots\\
r &= a_{0} + x_{0}b_{0} & {\color{red} \cdot 1}\\
\end{tabular}
$$

En additionnant toute les lignes les, les $b_{k}$ tombent

Donc les $b_{k}$ existent (uniques) et $r = P(x_{0})$

Ce processus est résumé dans le schéma de Hörner

 photo of table 

\paragraph{Exemple:}

Division euclidienne de $$P(x) = 4x^{3} + 2$$ par $$x+2$$
$x_{0} = -2$

 second table in photo 

Ainsi : 
$$4x^{3} + 2 = (4x^{2} - 8 + 16) \cdot (x+2) {\color{red}\underbrace{{\color{black}-30}}_{P(-2)}}$$


\paragraph{Corollaire:}

Le reste de la division de $P(x)$ par $x-x_{0}$ est $P(x_{0})$.

\paragraph{Définition:}

Soit $P \in \mathbb{R}[\ x\ ]$. $x_{0}$ est un zéro de $P$ (ou racine) si $P(x_{0})
= 0$.

\paragraph{Corollaire:}

$x_{0}$ est un zéro de $P(x)$ si et seulement si $P$ est divisible par $x -
x_{0}$.

\paragraph{Exemple:}

$$x_{0} = -1 $$ est racine évidente de $$P(x) = 3x^{3} - 2 x^{2} +4x + 9$$
$$P(-1) = 0$$ Alors $P(x)$ est divisible par $$x - x_{0} = x + 1$$


Pour trouver la factorisation: diviser ou utiliser le schéma de Hörner.

\paragraph{Définition:}

Soit $P$ un polynôme, $\deg P \geq 1$. Si $x_{0}$ est un zéro de $P$, il existe
$n \in \mathbb{N}^{*}$, appelé la multiplicité de $x_{0}$, tel que $$P(x) =
(x-x_{0})^{n} Q(x)$$
avec $$Q(x_{0}) \neq 0$$
et $$\deg Q = \deg P -n$$

\end{document}
