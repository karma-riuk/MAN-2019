%{{{-----------------------------------Basics----------------------------------%

%document class definition
\documentclass[
    11pt,
    a4paper,
    oneside,
    headinlcude, footinclude,
    twoside,
]{report}


%essential packages
\renewcommand*\rmdefault{ppl}
\usepackage[top=2.5cm,bottom=2.5cm,left=3cm,right=3cm]{geometry}
\usepackage[english]{babel} \usepackage[T1]{fontenc} \usepackage[utf8]{inputenc} \usepackage{xcolor} \usepackage{amssymb}


%additional packages
\usepackage{amssymb}
\usepackage{amsmath}
\usepackage[framemethod=Tikz]{mdframed}
\usepackage{tikz}
\usepackage{enumerate}
\usepackage{pgf,tikz,pgfplots} % for the transfer from geogebra to tikz
\usepackage{mathrsfs}% for the transfer from geogebra to tikz
\usepackage{graphicx}
\usepackage[makeroom]{cancel}
\usepackage{fancyhdr}
\usepackage{makecell} %to have thick hlines in the front page
\usepackage{tabularx}

%}}}

%{{{-----------------------------------Macros----------------------------------%

\newcommand{\myImplies}[0]{\rightarrow}

\newcommand{\powerset}[1]{\mathcal{P}(#1)}

\newcommand{\tvect}[3]{%
   \ensuremath{\Bigl(\begin{smallmatrix}#1\#2\#3\end{smallmatrix}\Bigr)}}

\newcommand{\myVector}[3]{\begin{pmatrix}#1\#2\#3\end{pmatrix}}

\newcommand{\tq}[0]{\ \textrm{ t.q. }\ }

\newcommand{\markDate}[1]{\begin{flushright}#1\end{flushright}}

\newcommand{\cqfd}[0]{\begin{flushright}$\Box$\end{flushright}}

\renewcommand{\vec}[1]{\overrightarrow{#1}}

\renewcommand{\to}[0]{\longrightarrow}

\renewcommand{\bar}[1]{\overline{#1}}

\def\getangle(#1)(#2)#3{
    \begingroup
        \pgftransformreset
        \pgfmathanglebetweenpoints{\pgfpointanchor{#1}{center}}{\pgfpointanchor{#2}{center}}
        \expandafter\xdef\csname angle#3\endcsname{\pgfmathresult}
    \endgroup
}

\newcommand\Warning{
    \makebox[1.4em][c]{
    \makebox[-5.5pt][c]{\raisebox{.2em}{!}}
    \makebox[0pt][c]{\color{red}\huge$\bigtriangleup$}}
}
%}}}

%{{{----------------------------------Settings---------------------------------%

\title{Maths 1A - AICC}

\author{Arnò Fauconnet}

\setlength{\parindent}{0pt} %disable initial indent on first paragraph of sections in the whole doc

% the style of the boxes
\newmdenv[
        roundcorner=10pt,
        middlelinecolor=red,
        backgroundcolor=gray!15,
        linewidth=2pt,
        frametitlerule=true]{highlightBox}

% increases the space between paragraphs
\setlength{\parskip}{.3em}

\usetikzlibrary{calc}
\usetikzlibrary{arrows}%for the transfer from geogebra to tikz
\pgfplotsset{compat=1.15}


% Geogebras wierd colors xD
\definecolor{ffffff}{rgb}{1.,1.,1.}
\definecolor{qqqqff}{rgb}{0.,0.,1.}
\definecolor{qqffqq}{rgb}{0.,1.,0.}
\definecolor{xfqqff}{rgb}{0.4980392156862745,0.,1.}
\definecolor{ffqqtt}{rgb}{1.,0.,0.2}
\definecolor{ududff}{rgb}{0.30196078431372547,0.30196078431372547,1.} 
\definecolor{ffqqqq}{rgb}{1.,0.,0.} 
\definecolor{xdxdff}{rgb}{0.49019607843137253,0.49019607843137253,1.}
\definecolor{zzttqq}{rgb}{0.6,0.2,0.}
\definecolor{uuuuuu}{rgb}{0.26666666666666666,0.26666666666666666,0.26666666666666666}
\definecolor{wwccff}{rgb}{0.4,0.8,1.}
\definecolor{qqttcc}{rgb}{0.,0.2,0.8}
\definecolor{ffwwzz}{rgb}{1.,0.4,0.6}
\definecolor{ttqqqq}{rgb}{0.2,0.,0.}

\tikzstyle{every node}=[font=\large]

\graphicspath{ {Maths_1A/figures/} }

% header and footer settings
\pagestyle{fancy}
\fancyhf{}
\fancyhead[LE,LO]{Arnaud Fauconnet}
\fancyhead[CE,CO]{\textsc{Maths 1A}}
\fancyhead[RE,RO]{MAN - Printemps 2019}
\fancyfoot[CE,CO]{\leftmark}
\fancyfoot[LE,RO]{\thepage}

\renewcommand{\headrulewidth}{2pt}
\renewcommand{\footrulewidth}{1pt}
%}}}

%%%%%%%%%%%%%%%%%%%%%%%%%%%%%%%%%%%%%%%%%%%%%%%%%%%%%%%%%%%%%%%%%%%%%%%%%%%%%%
%----------------------------------------------------------------------------%
%-------------------------------Text starts here-----------------------------%
%----------------------------------------------------------------------------%
%%%%%%%%%%%%%%%%%%%%%%%%%%%%%%%%%%%%%%%%%%%%%%%%%%%%%%%%%%%%%%%%%%%%%%%%%%%%%%


\begin{document}

\begin{titlepage}
   \begin{center}
       \vspace*{\fill}

       {\Huge EPFL}\\ 
%----------------------------------------------------------------------------%
       \vfill
       {\huge MAN}\\ [1em]
       {\Large Mise à niveau}\\
%----------------------------------------------------------------------------%
        \vfill
        \begin{tabularx}{\textwidth}{X}
            \Xhline{3\arrayrulewidth}\\
        \end{tabularx}\\ [2em]
        {\Huge Maths 1A} \\ [1em]
        \textsc{\huge Prepa-031(a)} \\ [2em]
       \begin{tabularx}{\textwidth}{X}
            \Xhline{3\arrayrulewidth}\\
        \end{tabularx}
%----------------------------------------------------------------------------%
        \vspace{.7cm}
        {\large
        \begin{tabularx}{.9\textwidth}{Xr}
            \textit{Student:} & \textit{Professor:}\\
            Arnaud \textsc{Fauconnet} & Guido \textsc{Burmeister}
        \end{tabularx}}
%----------------------------------------------------------------------------%
        \vfill
        {\Large Printemps - 2019}

%----------------------------------------------------------------------------%
        \vfill
        \includegraphics[width=7cm]{epfl-logo}

       \vfill
   \end{center} 
\end{titlepage} 
\setcounter{chapter}{1}
\chapter{Équations et inéquations sur les réels}
\label{sec:equations_et_inequations_sur_les_reels}

\section{Identité algébrique}

\paragraph{Propriétés:}
\begin{itemize}
    \item $(\mathbb{R}, +, \cdot)$ est un corps commutatif.
    \item L'identité remarquable. Soient $a, b \in \mathbb{R}$
        \begin{enumerate}
            \item $(a \pm b)^{2} = a^{2} \pm 2ab + b^{2}$ 
            \item $(a \pm b)^{3} = a^{3} \pm 3a^{2}b + 3ab^{2} \pm b^{3}$ 
            \item $a^{2} - b^{2} = (a - b) \cdot (a + b)$ {\color{red} $\quad a+b:$ expression conjugué de $a-b$  }
            \item $a^{3} - b^{3} = (a - b) \cdot (a^{2} + ab + b^{2})$ {\color{red} $\quad a^{2} + ab +b^{2}:$ expression conjugué de $a-b$}
            \item $a ^{n} - b^{n} = (a - b) \cdot (a^{n-1} + a^{n-2}b + ... + ab^{n-2} + b^{n-1})$ 
        \end{enumerate}
\end{itemize}

\paragraph{Exemples:} Amplifions par l'expression conjugué:

\begin{enumerate}
    \item $$\frac{1}{\sqrt{2}-1} {\color{red}\cdot \frac{\sqrt{2} + 1}{\sqrt{2}+1}}
        = \frac{\sqrt{2}+1}{2 - 1} = \sqrt{2}+1$$
    \item $$ \frac{1}{\sqrt[3]{x} + 1} {\color{red} \frac{(\sqrt[3]{x} - 
        \sqrt[3]{x} + 1)}{(\sqrt[3]{x} - \sqrt[3]{x} + 1)}} =
        \frac{(\sqrt[3]{x})^{2} + \sqrt[3]{x} + 1}{x + 1}, \quad (x \neq 1)$$
\end{enumerate}

\section{Ensemble solutions}
\label{sec:ensemble_solutions}

\paragraph{Exemples:}

\begin{enumerate}
    \item Résoudre en $x \in \mathbb{R}$:
        $$4x + 5 = 0$$
        L'unique solution est $x = - \frac{5}{4}$ \\
        L'ensemble solution est $S = \{- \frac{5}{4}\}$ 
    \item Résoudre en $x \in \mathbb{R}$:
        $$2x \geq 3$$
        L'ensemble solution $S = \left[\ \frac{3}{2}; + \infty\right[$ 
\end{enumerate}

\paragraph{Définition:}

Soient $f, g$ deux fonctions définies sur $D_{\text{déf}} \in \mathbb{R}$.

Résoudre l'équation:
$$f(x) = g(x)$$
ou l'inéquation $f(x) < g(x) (\text{strict})$ 

ou encore $f(x) \leq g(x) (\text{large})$ 

C'est chercher \textbf{l'ensemble aux valeurs} de $x$ vérifiant l'équation ou
l'inéquation

$$S = \{x \in \mathbb{D}_{ \text{déf}} \subset \mathbb{R} | \underbrace{x \text{ vérifie
l'équation ou l'inéquation}}_{ \text{proposition } P(x)}\}$$

\begin{itemize}
    \item $\mathbb{R}:$ l'ensemble des valeurs à considerer (référentiel)
    \item $\mathbb{D}_{\text{déf}}:$ l'ensemble des valeurs pour lesquelles l'expression
        $P(x)$ a un sens.
    \item L'équation ou l'inéquation est contrainte  ou propriété imposées
\end{itemize}

La résolution d'un problème passe par une succession de problème équivalents:
les ensembles solutions sont identiques.

\paragraph{Exemples:} Résoudre en $x \in \mathbb{R}$ 

\begin{enumerate}
    \item $P(x): \sqrt[3]{x} \leq 2$ 
        \begin{itemize}
            \item $\mathbb{D}_{\text{déf}} = \mathbb{R}$ 
            \item Équivalence: $$\underbrace{\sqrt[3]{x} \leq 2}_{\text{proposition }
                P(x), \text{ ensemble } A} \implies \underbrace{x \leq 2^{3} =
                8}_{\text{ proposition } Q(x), \text{ ensemble B}}$$ 

                Ainsi
            $$S = A = B =\ ]- \infty; 8\ ]$$
        \end{itemize}
    \item $P(x) : x^{2} = 64$ 
        \begin{itemize}
            \item $\mathbb{D}_{\text{déf}} = \mathbb{R}$ 
            \item Implication $$\underbrace{x^{2} = 64}_{P(x): A} \impliedby
                \underbrace{x = 8}_{Q(x): B}$$
        \end{itemize}
        On a $B = \{8\} \subset \{-8; 8\} = A = S$ 
    \item $P(x): \sqrt{x} = -4$ 
        \begin{itemize}
            \item $\mathbb{D}_{\text{déf}} = \mathbb{R}_{+}$ 
            \item Implication: $$\underbrace{\sqrt{x} = - 4}_{P(x): A}
                \implies \underbrace{x = (-4)^{2} = 16}_{Q(x): B}$$
                Ainsi $S = A = \emptyset \subset \{16\} = B \quad \quad$ 
                {\color{red} on a des solutions "parasites"}
        \end{itemize}
\end{enumerate}

\Warning Savoir (et énoncé) ce qu'on cherche (on veut faire) 

\subsection{Équations et inéquations linéaires}

\paragraph{Définition:}

Soient $a, b \in \mathbb{R}$ 
$$a \cdot x = b$$
est une équation linéaire en $x \in \mathbb{R}$ 

Clairement, $\mathbb{D}_{\text{déf}} = \mathbb{R}$\\
Pour résoudre une équation linéaire, \textbf{on cherche à isoler $x$:
discussion selon $a$} 

\begin{itemize}
    \item $a \neq 0$ (on peut diviser par $a$):
        $$ax=b \iff x = \frac{b}{a} \quad \text{ d'où } \quad S = \left\{\frac{b}{a}\right\}$$
    \item $a = 0$ (on ne peut pas diviser pas diviser par $a$ !)

        $$ax=b \iff 0x=b$$
        
        \begin{itemize}
            \item Si $b = 0: \quad 0x = 0.\quad$ Tout $x \in \mathbb{R}$ est
                solution $\quad S = \mathbb{R}$
            \item Si $b \neq 0: \quad 0x \neq 0. \quad$ Aucun $x$ est solution
                $S = \emptyset$
        \end{itemize}
\end{itemize}

\paragraph{Exemple:}

Résoudre en $x \in \mathbb{R}$ l'équation $$m^{2} \cdot x - m - 4x = 2$$ en
fonction de paramètre réel $m$ (pour chaque $m$ l'équation est différente)

\paragraph{Remarque:}

Équation du $1^{\text{er}}$ degré, en $x$, on cherche à \textbf{isoler $x$.} 
$$\underbrace{(m^{2} - 4)}_a x = \underbrace{m + 2}_b$$

Discussion du coefficient de $x$

\begin{itemize}
    \item $m^{2} - 4 \neq 0, \quad m \notin \{-2; 2\}$
        \[
            \begin{split}
            \implies x &= \frac{\cancel{m+2}}{(\cancel{m+2})\cdot(m-2)} = \frac{1}{m-2}\\
            \\
            S &= \left\{\frac{1}{m-2}\right\}
            \end{split}
        \]

    \item $m^{2} - 4 = 0, \quad m \in \{-2; 2\}$ 

        \begin{itemize}
            \item si $m = -2$
                \[
                    \begin{split}
                        (m^{2} - 4)x &= m + 2  \iff 0x = 0\\
                        S &= \mathbb{R}
                    \end{split}
                \]
            \item si $m = 2$
                \[
                    \begin{split}
                        (m^{2} - 4)x &= m + 2  \iff 0x = 4\\
                        S &= \emptyset
                    \end{split}
                \]
        \end{itemize}
\end{itemize}

\paragraph{Résumé:}

\begin{itemize}
    \item si $m \notin \{-2; 2\}, \quad S = \left\{\frac{1}{m-2}\right\}$
    \item si $m = -2  , \quad S = \mathbb{R}$
    \item si $m = 2 , \quad S = \emptyset$
\end{itemize}

\paragraph{Définition:}

Soient $a, b \in \mathbb{R}$
$$ax > b$$
est une inéquation \textbf{linéaire} en $x \in \mathbb{R}$: on cherche à
isoler $x$.

D'où une discussion de $a$.

\begin{itemize}
    \item $a > 0$:
        $$ax > b \iff x > \frac{b}{a}$$ $$S = \left] \frac{b}{a};
        + \infty \right[$$
    \item $a = 0$:
        $$ax > b \iff 0x > b$$
        \begin{itemize}
            \item si $b < 0$, tout $x$ est solution de $S$.
                $$S = \mathbb{R}$$
            \item si $b \geq 0$, aucun $x$ est solution de $S$.
                $$S = \emptyset$$
        \end{itemize}
    \item $a < 0$:
        $$ax > b \iff x < \frac{b}{a}$$ $$S = \left] - \infty;
         \frac{b}{a} \right[$$
\end{itemize}

\paragraph{Remarque:}

Résolution similaire pour $ax \geq b, ax < b \text{ et } ax \leq b$

\paragraph{Exemple:}

Résoudre en $x \in \mathbb{R}: m^{2}x - m - 4m \leq 2$ en fonction du
paramètre $m$.

\paragraph{Remarque:}

Inéquation linéaire, on cherche à isoler $x$ $$(m^{2} - 4)x \leq m + 2$$

Discussion du coefficient de $x$ 

\begin{itemize}
    \item Paramètre positif: $$m^{2} - 4 = (m-2)(m+2) > 0 \implies m \in \ ]- \infty; -2\ ]
        \cup [\ 2; + \infty[$$
            $$(m^{2} - 4) x \leq m + 2 \iff x \leq \frac{m + 2}{(m-2)(m+2)} =
            \frac{1}{m-2}$$
        $$S = \left] - \infty ; \frac{1}{m-2}\ \right]$$

    \item Paramètre nul: $$m^{2} - 4 = 0 \iff m \in \{-2; 2\}$$
        \begin{itemize}
            \item $m = -2$
                $$(m^{2} - 4) x \leq m + 2 \iff 0x \leq 0$$
                $$S = \mathbb{R}$$

            \item $m = 2$
                $$(m^{2} - 2) x \leq m + 2 \iff 0x \leq 4 $$
                $$S = \mathbb{R}$$
        \end{itemize}

    \item Paramètre négatif: $$m^{2} - 4 < 0 \iff m \in\ ] -2; 2\ [$$
            $$(m^{2} - 4) x \leq  m + 2 \iff x \geq \frac{1}{m+2}$$
            $$S = \left[\ \frac{1}{m-2}; + \infty \right[$$
\end{itemize}


\paragraph{Résumé}

\begin{itemize}
    \item si $ m  \in\ ]-\infty; -2\ [\ \cup\ ]2; + \infty[, \quad S =
            \left]-\infty; \frac{1}{m-2}\ \right]$
    \item si $ m \in \{-2; 2\}, \quad S = \mathbb{R}$
    \item si $ m  \in\ ]-2; 2\ [, \quad S = \left[\frac{1}{m-2}; + \infty\right[$
\end{itemize}

\section{Équations et inéquations rationelles}
\label{sec:equations_et_inequations_rationelles}

\paragraph{Définition:}

Une fonction rationnelle en $x \in \mathbb{R}$ est un quotient de fonction
polynomiale. Pour résoudre une équation $f(x) = g(x)$ ou inéquations $f(x) <
g(x)$ sur la fonction rationnelle.

\begin{itemize}
    \item On différencie le domaine de définition $\mathbb{D}_{\text{déf}}$.
    \item On passe \textbf{toutes} les expression du même côté de l'égalité (ou
        inégalités) et on étudie le signe en factorisant.
\end{itemize}


\paragraph{Exemple:}

Résoudre en $x$ l'inéquation  $x > \frac{4}{x}$

\begin{itemize}
    \item $\mathbb{D}_{\text{déf}} = \mathbb{R}$
    \item Inéquation rationnelle: on porte toute du même côté
        \[
            \begin{split}
                x - \frac{4}{x} > 0 &\iff \frac{x^{2} - 4}{x} > 0\\
                &\iff \frac{(x + 2)\cdot(x-2)}{x} > 0
            \end{split}
        \]

        \pagebreak

    \item tableau des signes (remarque: le valeurs remarquables sont $-2, 0, 2$)

\end{itemize}

\begin{center}
    
    \begin{tabular}{c|ccccccc}
        $x$ &  & $-2$ &  & $0$ &  & $2$  & \\
        &  & {\scriptsize$\downarrow$} &  & {\scriptsize$\downarrow$} &  &{\scriptsize$\downarrow$}  & \\
        \hline
        $x + 2$ & $-$ & $0$  & $+$  & $+$  & $+$  & $+$  & $+$ \\
        $x - 2$ & $-$  & $-$  & $-$  & $-$  & $-$  & $0$  & $+$ \\
        $x$ & $-$  & $-$  & $-$  & $0$  & $+$  & $+$  & $+$ \\
        \hline
        $\frac{(x+2)\cdot(x-2)}{x}$ & $-$  & $0$  & $+$  & \fcolorbox{gray}{gray}{\makebox[.4em][c]{\makebox[0pt][c]{\raisebox{-.1em}{\color{gray}0}}} }  & $-$  & $0$  & $+$ \\
    \end{tabular}
\end{center}

$$S =\ ]-2; 0[\ \cup\ ] 2; +\infty[$$



 missing stuff


\section{sectoin missing}
\label{sec:sectoin_missing}
 missing stuff


\section{sectoin missing}
\label{sec:sectoin_missing}

 missing stuff


\section{Valeur absolue}
\label{sec:valeur_absolue}

\paragraph{Définition:}

Soit $x \in \mathbb{R}$. La valeur absolue de $x$, not\'ee $|x|$ est r\'eel
positif ou null
$$ |x| = \left\{
    \begin{array}{ll}
    x & \text{ si } x \geq 0\\
    -x & \text{ si } x < 0\\
    \end{array}
\right. $$

\paragraph{Exemple:}

$$|-3| = - (-3) = 3 \quad \quad \quad |\sqrt{2}| = \sqrt{2}$$

\paragraph{Propriétés:}

Soient $x, y \in \mathbb{R}$. Alors 
\begin{enumerate}
    \item $|x| \geq 0$
    \item $|x| = 0 \iff x = 0$
    \item $|x^{2}| = x^{2}$
    \item $|-x| = |x|$
    \item $x = \text{sgn}(x) \cdot |x| \text{ et } |x| = \text{sgn} (x) \cdot x$
    \item $-|x| \leq x \leq |x|$
    \item $|x + y| \leq |x| + |y| \text{(inégalité triangulaire)}$
    \item $|x\cdot y| = |x| \cdot |y|$
\end{enumerate}


\subsection{Equation \`a valeur absolue}

\paragraph{Remarque:}

L'\'equation $|x| = a, \quad a \in \mathbb{R}$ ne peut clairement pas avoir de
solutions en $x \in \mathbb{R}$ si $a < 0$

\paragraph{Théorème:}

On a l'\'equivalence $$|x| = a \iff a \geq 0 \text{ et }  \left\{
    \begin{array}{l}
    x = a\\
    x = -a\\
    \end{array}
\right.$$


En effet, 

\begin{center}
    \begin{minipage}{.49\linewidth}
        \resizebox{\textwidth}{!}{
            \begin{tikzpicture}[line cap=round,line join=round,>=triangle 45,x=1.0cm,y=1.0cm]
            \begin{axis}[
            x=1.0cm,y=1.0cm,
            axis lines=middle,
            xmin=-6.0,
            xmax=6.0,
            ymin=-2.5,
            ymax=6.0,
            xtick={-5.0,...,5.0},
            ytick={-2.0,-1.0,...,5.0},
            xticklabels={},
            yticklabels={},
            xlabel={$x$},
            ylabel={$y$},
            ]
            \clip(-6.,-2.5) rectangle (6.,6.);
            \draw[line width=0.4pt,color=qqqqff,smooth,samples=100,domain=-6.0:6.0] plot(\x,{abs((\x))});
            \draw [line width=0.4pt,color=ffqqqq,domain=-6.:6.] plot(\x,{(--4.-0.*\x)/1.});
            \draw [line width=0.4pt,dash pattern=on 3pt off 3pt,color=ffqqqq] (-4.,0.) node [anchor = north]{$-a$}-- (-4.,4.);
            \draw [line width=0.4pt,dash pattern=on 3pt off 3pt,color=ffqqqq] (4.,0.) node [anchor = north]{$a$}-- (4.,4.);
            \begin{scriptsize}
                \draw[color=qqqqff] (5.3,6) node [anchor =north east]{$y = |x|$};
                \draw[color=ffqqqq] (6,4) node [anchor =south east]{$y = a$};
                \draw[color=ffqqqq] (-3,-1) node {$a > 0$};
            \end{scriptsize}
            \end{axis}
            \end{tikzpicture}
        }
        $$S = \{-a; a\}$$
    \end{minipage}
    \begin{minipage}{.49\linewidth}
        \resizebox{\textwidth}{!}{
            \begin{tikzpicture}[line cap=round,line join=round,>=triangle 45,x=1.0cm,y=1.0cm]
            \begin{axis}[
            x=1.0cm,y=1.0cm,
            axis lines=middle,
            xmin=-6.0,
            xmax=6.0,
            ymin=-2.5,
            ymax=6.0,
            xtick={-5.0,...,5.0},
            ytick={-2.0,-1.0,0.0,...,5.0},
            xticklabels={},
            yticklabels={},
            xlabel={$x$},
            ylabel={$y$},
            ]
            \clip(-6.,-2.5) rectangle (6.,6.);
            \draw[line width=0.4pt,color=qqqqff,smooth,samples=100,domain=-6.0:6.0] plot(\x,{abs((\x))});
            \draw [line width=0.4pt,color=ffqqqq,domain=-6.:6.] plot(\x,{(-1.5-0.*\x)/1.});
            \begin{scriptsize}
                \draw[color=qqqqff] (5.3,6) node [anchor =north east]{$y = |x|$};
                \draw[color=ffqqqq] (6,-1.5) node [anchor =south east]{$y = a$};
                \draw[color=ffqqqq] (-3,-1) node {$a < 0$};
            \end{scriptsize}
            \end{axis}
            \end{tikzpicture}
        }
        $$S = \emptyset$$
    \end{minipage}
\end{center}

\paragraph{Remarque:}

(g\'en\'eralisation)

Soient $f$ et $g$ deux fonctions r\'elles. On a l'\'equivalence 

\begin{highlightBox}
    $$|f(x)| = g(x) \iff g(x) \geq 0 \text{ et }  \left\{
        \begin{array}{l}
        f(x)  = g(x)\\
        f(x)  = -g(x)\\
        \end{array}
    \right.$$
\end{highlightBox}

\paragraph{Remarque:}

On ne discute pas le signe de $f(x)$, mais seulement celui de $g(x)$
(condition de positivit\'e). On travaille donc sur le r\'ef\'erentiel
restreint $\mathbb{D}_{\text{d\'ef}} \cap \mathbb{D}_{\text{positif}}$

\paragraph{Exemple:}

R\'esoudre en $x \in \mathbb{R}$

$$| x^{2} + 2x - 5 | = x + 1$$

\begin{itemize}
    \item domaine de d\'efinition: $\mathbb{D}_{\text{d\'ef}} = \mathbb{R}$
    \item \'equivalence:
        $$| x^{2} + 2x - 5| = x + 1 \iff x + 1 \geq 0 \text{ et } \left\{
            \begin{array}{lc}
            x^{2} + 2 x - 5 = x+1 & (1)\\
            x^{2} + 2 x - 5 = -(x+1) & (2)\\
            \end{array}
        \right.$$

    \item condition de positivit\'e : $x + 1 \geq 0$

        D'o\`u $\mathbb{D}_{\text{pos}} = [\ -1; + \infty\ ]$

    \item \'equatoin $(1)$
        \[
            \begin{split}
               (1):& x^{2} + x - 6 = 0     \\
               &(x+3) \cdot (x-2) = 0
            \end{split}
        \]
        d'o\`u $S_{1} = \{-3, 2\}$

        \paragraph{Remarque:}
        
        $-3$ est \`a exclure: $-3 \notin \mathbb{D}_{\text{pos}}$

    \item l'\'equation $(2)$
        \[
            \begin{split}
               (2):& x^{2} + 3x - 4 = 0     \\
               &(x+4) \cdot (x-1) = 0
            \end{split}
        \]
        d'o\`u $S_{1} = \{-4, 1\}$

    \item Solution: $$S = \mathbb{D}_{\text{d\'ef}} \cap \mathbb{D}_{\text{pos}} \cap (S_{1}
        \cup S_{2}) = \{1, 2\}$$

\end{itemize}

\subsection{In\'equation \`a valeur absolue}
\label{sub:in'equation_`a_valeur_absolue}

\paragraph{Remarque:}

L'in\'equation $|x| \leq a, \quad a \in \mathbb{R}$, ne peut pas avoir de
solution si $a < 0$. On n'a pourtant besoin de discuter le signe de $a$!

\paragraph{Théorème:}

Soit $a \in \mathbb{R}$. On a l'\'equivalence

$$|x| \leq a \iff \left\{
    \begin{array}{l}
    x \leq a\\
    \text{et } \\
    x \geq -a\\
    \end{array}
\right.$$

En effet,

\begin{center}
    \begin{minipage}{.49\linewidth}
        \resizebox{\textwidth}{!}{
            \begin{tikzpicture}[line cap=round,line join=round,>=triangle 45,x=1.0cm,y=1.0cm]
            \begin{axis}[
            x=1.0cm,y=1.0cm,
            axis lines=middle,
            xmin=-6.0,
            xmax=6.0,
            ymin=-2.5,
            ymax=6.0,
            xtick={-5.0,...,5.0},
            ytick={-2.0,-1.0,...,5.0},
            xticklabels={},
            yticklabels={},
            xlabel={$x$},
            ylabel={$y$},
        ]
            \clip(-6.,-2.5) rectangle (6.,6.);
            \draw[line width=4.pt] (-7.255053904054706,10.395884435575802) -- (-3.0274049107279426,10.395884435575802);
            \draw [line width=0.4pt,color=ffqqqq,domain=-6.:6.] plot(\x,{(--4.-0.*\x)/1.});
            \draw[line width=0.4pt,color=qqqqff,smooth,samples=100,domain=-6.0:6.0] plot(\x,{abs((\x))});
            \draw [line width=0.4pt,dash pattern=on 2pt off 2pt,color=ffqqqq] (4.,0.)node [anchor = north] {$a$}-- (4.,4.);
            \draw [line width=0.4pt,dash pattern=on 2pt off 2pt,color=ffqqqq] (-4.,0.)node [anchor = north] {$-a$}-- (-4.,4.);
            \draw [line width=0.4pt,color=ffqqqq] (-4.,-0.5)-- (4.,-0.5);
            \draw [line width=0.4pt,color=qqffqq,domain=-6.0:4.0] plot(\x,{(--3.984669879181435-0.*\x)/-4.});
            \draw [line width=0.4pt,color=xfqqff,domain=-4.0:6.0] plot(\x,{(-6.014250288791247-0.*\x)/4.});
            \begin{scriptsize}
            \draw[color=ffqqqq] (4, -0.5) node [anchor=west] {$S$};
            \draw[color=qqffqq] (4, -1) node [anchor=west]{$x \leq a$};
            \draw[color=xfqqff] (-4, -1.5) node [anchor=east]{$x \geq -a$};
            \draw[color=qqqqff] (5.3,6) node [anchor =north east]{$y = |x|$};
            \draw[color=ffqqqq] (6,4) node [anchor =south east]{$y = a$};
            \draw[color=ffqqqq] (-3,5) node {$a > 0$};
            \end{scriptsize}
            \end{axis}
            \end{tikzpicture}
        }
        $$S = [\ -a, a\ ] = {\color{qqffqq}] - \infty; a\ ]} \cap {\color{xfqqff}[\ -a; + \infty[}$$
    \end{minipage}
    \begin{minipage}{.49\linewidth}
        \resizebox{\textwidth}{!}{
            \begin{tikzpicture}[line cap=round,line join=round,>=triangle 45,x=1.0cm,y=1.0cm]
            \begin{axis}[
            x=1.0cm,y=1.0cm,
            axis lines=middle,
            xmin=-6.0,
            xmax=6.0,
            ymin=-2.5,
            ymax=6.0,
            xtick={-5.0,...,5.0},
            ytick={-2.0,-1.0,0.0,...,5.0},
            xticklabels={},
            yticklabels={},
            xlabel={$x$},
            ylabel={$y$},
            ]
            \clip(-6.,-2.5) rectangle (6.,6.);
            \draw[line width=0.4pt,color=qqqqff,smooth,samples=100,domain=-6.0:6.0] plot(\x,{abs((\x))});
            \draw [line width=0.4pt,dash pattern=on 2pt off 2pt,color=ffqqqq] (1.,0.) node [anchor= south] {$-a$}-- (1.,-1.);
            \draw [line width=0.4pt,dash pattern=on 2pt off 2pt,color=ffqqqq] (-1.,0.) node [anchor= south] {$a$}-- (-1.,-1.);
            \draw [line width=0.4pt,color=ffqqqq,domain=-6.:6.] plot(\x,{(-1-0.*\x)/1.});
            \draw [line width=0.4pt,color=qqffqq,domain=-6.0:-0.996892396320315] plot(\x,{(--5.3142074179149255-0.*\x)/-3.553949038688488});
            \draw [line width=0.4pt,color=xfqqff,domain=1.0028089788665404:6.0] plot(\x,{(-2.703447941698279-0.*\x)/1.807967860108991});
            \begin{scriptsize}
                \draw[color=qqqqff] (5.3,6) node [anchor =north east]{$y = |x|$};
                \draw[color=ffqqqq] (6,-1) node [anchor =south east]{$y = a$};
                \draw[color=ffqqqq] (-3,5) node {$a < 0$};
                \draw[color=qqffqq] (-1, -1.5) node [anchor=north east]{$x \leq a$};
                \draw[color=xfqqff] (1, -1.5) node [anchor=north west]{$x \geq -a$};
            \end{scriptsize}
            \end{axis}
            \end{tikzpicture}
        }
        $$S =  \emptyset = {\color{qqffqq}]- \infty; a\ ]} \cap {\color{xfqqff}[\ -a; + \infty[}$$
    \end{minipage}
\end{center}



\paragraph{Théorème:}

Soient $f$ et $g$ deux fonctions r\'eelles. On a l'\'equivalence 

\begin{highlightBox}
    $$|f(x)| \leq g(x)  \iff \left\{
        \begin{array}{l}
        f(x) \leq g(x)\\
        \text{et } \\
        f(x) \geq -g(x)\\
        \end{array}
    \right.$$
\end{highlightBox}

\paragraph{Remarques:}

\begin{enumerate}
    \item On ne discutera pas le signe de $f(x)$, ni celui de $g(x)$ (le cas
        trivial $g(x)< 0$ est rejet\'e lors de l'intersection)
    \item Idem avec l'in\'egualit\'e stricte.
\end{enumerate}


\paragraph{Remarque:}

L'in\'equation $|x| \leq a, \quad a \in \mathbb{R}$ admet clairement tout $x
\in \mathbb{R}$ comme solution si $a < 0$ (une valeur absolue est toujours
grand qu'un nombre n\'egatif).  On ne discutera pourtant pas le signe de $a$!
 
$$|x| \leq a, \quad a < 0 \text{ trivial } : \quad S = \emptyset$$
$$|x| \geq a, \quad a < 0 \text{ trivial } : \quad S = \mathbb{R}$$

\paragraph{Théorème:}

Soit $a \in \mathbb{R}$. On a l'\'equivalence

$$|x| \geq a \iff \left\{
    \begin{array}{l}
    x \geq a\\
    \text{ou } \\
    x \leq -a\\
    \end{array}
\right.$$

En effet, 

\begin{center}
    \begin{minipage}{.49\linewidth}
        \resizebox{\textwidth}{!}{
            \begin{tikzpicture}[line cap=round,line join=round,>=triangle 45,x=1.0cm,y=1.0cm]
            \begin{axis}[
            x=1.0cm,y=1.0cm,
            axis lines=middle,
            xmin=-6.0,
            xmax=6.0,
            ymin=-2.5,
            ymax=6.0,
            xtick={-5.0,...,5.0},
            ytick={-2.0,-1.0,...,5.0},
            xticklabels={},
            yticklabels={},
            xlabel={$x$},
            ylabel={$y$},
        ]
            \clip(-6.,-2.5) rectangle (6.,6.);
            \draw[line width=4.pt] (-7.255053904054706,10.395884435575802) -- (-3.0274049107279426,10.395884435575802);
            \draw [line width=0.4pt,color=ffqqqq,domain=-6.:6.] plot(\x,{(--4.-0.*\x)/1.});
            \draw[line width=0.4pt,color=qqqqff,smooth,samples=100,domain=-6.0:6.0] plot(\x,{abs((\x))});
            \draw [line width=0.4pt,dash pattern=on 2pt off 2pt,color=ffqqqq] (4.,0.)node [anchor = north] {$a$}-- (4.,4.);
            \draw [line width=0.4pt,dash pattern=on 2pt off 2pt,color=ffqqqq] (-4.,0.)node [anchor = north] {$-a$}-- (-4.,4.);
            \draw [line width=0.4pt,color=qqffqq,domain=-6.0:-4.0] plot(\x,{(--3.984669879181435-0.*\x)/-4.});
            \draw [line width=0.4pt,color=xfqqff,domain=4.0:6.0] plot(\x,{(-6.014250288791247-0.*\x)/4.});
            \draw [line width=0.4pt,color=ffqqqq,domain=-6.0:-4.0] plot(\x,{(0*\x-.5)});
            \draw [line width=0.4pt,color=ffqqqq,domain=4.0:6.0] plot(\x,{(0*\x-.5)});
            \begin{scriptsize}
            \draw[color=ffqqqq] (4, -0.5) node [anchor=east] {$S$};
            \draw[color=qqffqq] (-4, -1) node [anchor=west]{$x \leq -a$};
            \draw[color=xfqqff] (4, -1.5) node [anchor=east]{$x \geq a$};
            \draw[color=qqqqff] (5.3,6) node [anchor =north east]{$y = |x|$};
            \draw[color=ffqqqq] (6,4) node [anchor =south east]{$y = a$};
            \draw[color=ffqqqq] (-3,5) node {$a > 0$};
            \end{scriptsize}
            \end{axis}
            \end{tikzpicture}
        }
        $$S = {\color{green}]-\infty; -a\ ]} \cup {\color{xfqqff}[\ a; + \infty\ [}$$
    \end{minipage}
    \begin{minipage}{.49\linewidth}
        \resizebox{\textwidth}{!}{
            \begin{tikzpicture}[line cap=round,line join=round,>=triangle 45,x=1.0cm,y=1.0cm]
            \begin{axis}[
            x=1.0cm,y=1.0cm,
            axis lines=middle,
            xmin=-6.0,
            xmax=6.0,
            ymin=-2.5,
            ymax=6.0,
            xtick={-5.0,...,5.0},
            ytick={-2.0,-1.5,...,5.0},
            xticklabels={},
            yticklabels={},
            xlabel={$x$},
            ylabel={$y$},
            ]
            \clip(-6.,-2.5) rectangle (6.,6.);
            \draw[line width=0.4pt,color=qqqqff,smooth,samples=100,domain=-6.0:6.0] plot(\x,{abs((\x))});
            \draw [line width=0.4pt,dash pattern=on 2pt off 2pt,color=ffqqqq] (1.,0.) node [anchor= south] {$-a$}-- (1.,-.5);
            \draw [line width=0.4pt,dash pattern=on 2pt off 2pt,color=ffqqqq] (-1.,0.) node [anchor= south] {$a$}-- (-1.,-.5);
            \draw [line width=0.4pt,color=ffqqqq,domain=-6.:6.] plot(\x,{(-.5-0.*\x)/1.});
            \draw [line width=0.4pt,color=qqffqq,domain=-1:6] plot(\x,{(-1-0.*\x)});
            \draw [line width=0.4pt,color=xfqqff,domain=1:-6] plot(\x,{(-2.703447941698279-0.*\x)/1.807967860108991});
            \begin{scriptsize}
                \draw[color=qqqqff] (5.3,6) node [anchor =north east]{$y = |x|$};
                \draw[color=ffqqqq] (6,-1) node [anchor =south east]{$y = a$};
                \draw[color=ffqqqq] (-3,5) node {$a < 0$};
                \draw[color=qqffqq] (-1, -1) node [anchor=east]{$x \geq a$};
                \draw[color=xfqqff] (1, -1.5) node [anchor=west]{$x \leq -a$};
            \end{scriptsize}
            \end{axis}
            \end{tikzpicture}
        }
        $$S = \mathbb{R} = {\color{xfqqff}]-\infty; -a\ ]} \cup {\color{green}[\ a; + \infty\ [}$$
    \end{minipage}
\end{center}


\paragraph{Théorème:}

Soient $f$ et $g$ deux fonctions r\'eelles. On a l'\'equilavence 

\begin{highlightBox}
    $$|f(x)| \geq g(x) \iff \left\{
        \begin{array}{l}
        f(x) \geq g(x)\\
        \text{ou } \\
        f(x) \leq -g(x)\\
        \end{array}
    \right.$$
\end{highlightBox}

\paragraph{Remarques:}

\begin{enumerate}
    \item On ne discutera ni le signe de $f(x)$, ni celui de $g(x)$. Le cas
        trivial $g(x) < 0$ est trait\'e par la r\'eunion.
    \item idem pour l'in\'egalit\'e stricte. 
\end{enumerate}

\paragraph{Exemple:}

R\'esoudre en $x \in \mathbb{R}$

$$|x| + \frac{x-1}{2} < 0$$

\begin{itemize}
    \item domaine de d\'efinition: $\mathbb{D}_{\text{d\'ef}} = \mathbb{R}$

    \item \'equivalence 

        $$|x| < - \frac{x-1}{2} \iff \left\{
            \begin{array}{lc}
            x < - \frac{x-1}{2} & (1) \\
            \text{et } \\
            x > \frac{x-1}{2} & (2) \\
            \end{array}
        \right.$$

    \item in\'equation $(1)$. On isole $x$ 
        $$3 x < 1 \text{ d'o\'u } S_{1} = \left]-\infty; \frac{1}{3} \right[$$
    \item in\'equation $(1)$. On isole $x$ 
        $$ x > -1 \text{ d'o\'u } S_{2} = ]-1; + \infty[$$

    \item Solution:
        $$S = \mathbb{D}_{\text{d\'ef}} \cap S_{1} \cap S_{2} = \left]-1, \frac{1}{3}\right[$$
\end{itemize}

\paragraph{Exemple:}

R\'esoudre en $x \in \mathbb{R}$
$$|x - 2| > \frac{2x-4}{x}$$

\begin{itemize}
    \item $\mathbb{D}_{\text{d\'ef}} = \mathbb{R}^{*}$

    \item $$|x -2| < \frac{2x-4}{x} \iff \left\{
        \begin{array}{lc}
        x - 2 > \frac{2x - 4}{x} & (1)\\
        \text{ou } \\
        x - 2 < -\frac{2x - 4}{x} & (2)\\
        \end{array}
    \right.$$

    \item in\'equation $(1)$ 

        \[
            \begin{split}
               x - 2 - \frac{2x-4}{x}&> 0 \\    
               (x-2)\cdot \left(1 - \frac{2}{x}\right) &> 0\\
               \frac{(x-2)^{2}}{x} &>0
            \end{split}
        \]
        d'o\`u $$S_{1} = \mathbb{R}^{*}_{+} \backslash \{2\} = ]0; 2[\ \cup \
        ] 2 ; +\infty[$$

    \item in\'equation $(2)$ 

        \[
            \begin{split}
               x - 2 + \frac{2x-4}{x}&< 0 \\    
               (x-2)\cdot \left(1 + \frac{2}{x}\right) &< 0\\
               \frac{(x-2)\cdot (x+2)}{x} &<0
            \end{split}
        \]
        d'o\`u $$S_{1} = \mathbb{R}^{*}_{+} \backslash \{2\} = ]0; 2[\ \cup \
        ] 2 ; +\infty[$$


        \begin{center}
            \begin{tabular}{c|ccccccc}
                $x$ &  & $-2$ &  & $0$ &  & $2$  & \\
                \hline
                $x - 2$ & $-$  & $-$  & $-$  & $-$  & $-$  & $0$  & $+$ \\
                $x + 2$ & $-$ & $0$  & $+$  & $+$  & $+$  & $+$  & $+$ \\
                $x$ & $-$  & $-$  & $-$  & $0$  & $+$  & $+$  & $+$ \\
                \hline
                $\frac{(x+2)\cdot(x-2)}{x}$ & $-$  & $0$  & $+$  & \fcolorbox{gray}{gray}{\makebox[.4em][c]{\makebox[0pt][c]{\raisebox{-.1em}{\color{gray}0}}} }  & $-$  & $0$  & $+$ \\
            \end{tabular}
        \end{center}
        d'o\`u $$S_{2} = ] -\infty; -2[\ \cup \ ] 0; 2[$$

    \item Solution:
        $$S = \mathbb{D}_{\text{d\'ef}} \cap (S_{1} \cup S_{2})$$
    $$s = ]-\infty; -2[\ \cup\ ]\ 0; 2\ [\ \cup\ ]2; +\infty[$$

\end{itemize}

\section{Racines}
\label{sec:racines}

\subsection{Racines positives (ou arithmétique)}
\label{sub:racines_positives_ou_arithmetique_}

Soit $n \in \mathbb{N}^{*}$

Graphe de $x^{n}$

% fig 1

\paragraph{Définition:}

Soient $a \in \mathbb{R}_{+}$ et $n \in \mathbb{N}^{*}$

Le \textbf{réel positif} $x$ vérifiant $x^{n} = a$ est appelé $n^{\text{e}}$
racine positive de $a$.

\paragraph{Exemple:}

$\sqrt{-4}$ $n$ est pas définie

$\sqrt[3]{-27} = 3$: racine cubique 

$-2$ n'est pas une racine de $4$.

\paragraph{Propriétés:}

Soient $a, b \in \mathbb{R}_{+}, \quad m,n \in \mathbb{N}^{*}$

\begin{enumerate}
    \item $(\sqrt[n]{a})^{n}  = a$
    \item $\sqrt[n]{a^{m}}  = (\sqrt[n]{a})^{m}$
    \item $\sqrt[n]{a \cdot b} = \sqrt[n]{a} \cdot \sqrt[n]{b}$
    \item $\sqrt[m]{\sqrt[n]{a}} = \sqrt[n\cdot m]{a}$
\end{enumerate}

\paragraph{Remarque:}

Ce sont les même règles que celle puissances en posant 



$$\sqrt[q]{a^{p}} = a ^{\frac{p}{q}}, \quad a \in \mathbb{R}^{*}_{-}, p \in \mathbb{Z},
q \in \mathbb{N}^{*}$$

\paragraph{Exemple:}

$$7^{-\frac{2}{3}} = \frac{1}{7^{\frac{2}{3}}} = \frac{1}{\sqrt[3]{7^{2}}}$$
$$\sqrt{3x^{2}} = \sqrt{3} \cdot \sqrt{x^{2}} = \sqrt{3} \cdot |x|$$

\subsection{Racines réelles (ou zéros)}
\label{sub:racines_reelles_ou_zeros_}

\paragraph{Définition:}

Soient $a \in \mathbb{R}$ et $n \in \mathbb{N}^{*}$. Un nombre $x \in \mathbb{R}$
vérifiant $x^{n} = a$ est une $n^{\text{e}}$ racine \textbf{réelle} de $a$.

\paragraph{Exemple:}

\begin{itemize}
    \item $2$ et $-2$ sont les solutions à $$x^{2} =4$$ 
        Ce sont les $2$ racines carrées réelles de $4$.
    \item $-3$ est racine cubique réelle de $-27$. En effet $$(-3)^{3} = -27$$
\end{itemize}

Discussion graphique de l'équation en $x$ $$x^{n} = a$$

\begin{itemize}
    \item  $n$ pair
        \begin{center}
            \begin{minipage}{.5\linewidth}
                \resizebox{\textwidth}{!}{
%fig 2
                }
            \end{minipage}
            \begin{minipage}{.49\linewidth}
                \setlength{\parskip}{.3em}
                \paragraph{Remarque:}
                Axe de symétrie en $x = 0$
                \begin{itemize}
                    \item si $a > 0$: $2$ racines distinctes:
                        $$S = \{-\sqrt[n]{a}, +\sqrt[n]{a}\}$$
                    \item si $a = 0$: racine double:
                        $$S = \{0\}$$
                    \item si $a < 0$: pas de racines
                        $$S = \emptyset$$
                \end{itemize}
            \end{minipage}
        \end{center}
    \item $n$ impair
        \begin{center}
            \begin{minipage}{.5\linewidth}
                \resizebox{\textwidth}{!}{
%fig 3

                }
            \end{minipage}
            \begin{minipage}{.49\linewidth}
                \setlength{\parskip}{.3em}
                \paragraph{Remarque:}
                Centre de symétrie à l'origine.  $\forall a \in \mathbb{R}$,
                il y a une unique solution

                \paragraph{Remarque:}
                Pour un $n$ impair, on admet l'écriture $$-\sqrt[n]{-a} = \sqrt[n]{a}$$
                d'où $$S=\left\{\sqrt[n]{a}\right\}$$
            \end{minipage}
        \end{center}
\end{itemize}

\paragraph{Exemple:}

$$
\left.
    \begin{array}{rl}
    x^{4} = 16 & \text{ exposant $n = 4$  pair}\\
     & 16 > 0\\
    \end{array}
\right\} 2 \text{ solution distinctes }
$$
$$S = \{-\sqrt[4]{16}, \sqrt[4]{16}\} = \{-2, 2\}$$

\paragraph{Exemple:}

$$x^{3} + 8 = 0 \iff x^{3} = -8\ (\text{exposant $n = 3$ impair})$$
$$\implies S = \{\sqrt[3]{-8}\} = \{-2\}$$

\paragraph{Conséquences:}

\begin{itemize}
    \item Soit $a \in \mathbb{R}^{*}$. Alors 
        $$
        \sqrt[n]{a^{n}} = 
        \left\{
            \begin{array}{ll}
                a & \text{ si $n$ impair }\\
                |a| & \text{ si $n$ est pair}\\
            \end{array}
        \right.
        $$

    \item Soient $a, b \in \mathbb{R}_{+}$ (condition de positivité). Alors 
        \[
            \begin{split}
                a = b &\iff a^{n} = b^{n} \quad (n \in \mathbb{N}^{*})\\
                a < b &\iff a^{n} < b^{n}
            \end{split}
        \]

    \item Soient $a, b \in \mathbb{R}$ et $n$ impair. Alors 
        \[
            \begin{split}
                a = b &\iff a^{n} = b^{n}\\
                a < b &\iff a^{n} < b^{n}\\
            \end{split}
        \]
        (car $x^{na}$ est strictement croissante)
\end{itemize}

\subsection{Équations et inéquations rationnelles}

Dans une équation / inéquation du type
$$\sqrt{f} = g \quad \sqrt{f} < g \quad \sqrt{f} > g$$

on "veut élever au carré" pour faire tomber la $\sqrt{}$ % TODO look up how to do a sime \sqrt sign

C'est en ordre si $\sqrt{f} \geq 0$ (c'est le cas!) et $g \geq 0$ (condition
de positivité). D'où la discussion du signe de $g$.

\paragraph{Théorème:}

Soient $f$ et $g$ 2 fonctions réelles.

Sur le domaine de définition (dont la condition $f(x) > 0$), on a
l'équivalence 
\begin{highlightBox}
    $$\sqrt{f(x)} = g(x) \iff g(x) \geq 0 \quad \text{ et } \quad f(x) = g^{2}(x)$$
\end{highlightBox}

En effet si $g(x) < 0$, il n'y a pas de solution, car $\sqrt{f(x)} > 0$

\paragraph{Théorème:}

Soient $f, g$ 2 fonctions réelles sur $\mathbb{D}_{\text{déf}}$, on a
l'équivalence 

\begin{highlightBox}
    $$\sqrt{f(x)} \leq g(x) \iff g(x) \geq 0 \quad \text{ et } \quad f(x) \leq
    g^{2}(x)$$
\end{highlightBox}

En effet, si $g(x)< 0$, il n'y a pas de solution.

\paragraph{Théorème:}

Soient $f, g$ 2 fonctions réelles sur $\mathbb{D}_{\text{déf}}$ on a
l'équivalence 

\begin{highlightBox}
    $$\sqrt{f(x)} > g(x) \iff 
    \left\{
        \begin{array}{l}
        g(x) < 0\\
        \text{ou}\\
        g(x) \geq 0 \text{ et } f(x) > g^{2}(x)
        \end{array}
    \right.$$
\end{highlightBox}

En effet si $g(x) < 0$, tout $x$ est solution :
$$\sqrt{f(x)} > 0 > \underbrace{g}_{<0}$$

\paragraph{Exemple:}

\begin{enumerate}
    \item Résoudre en $x \in \mathbb{R}$ $$\sqrt{x^{2} - 3x + 6} = 4x - 6$$

    \begin{itemize}
        \item $\mathbb{D}_{\text{déf}}: x^{2}-3x + 6 \geq 0$ 
            
            comme $\Delta < 0, \mathbb{D}_{\text{déf}} = \mathbb{R}$
        \item Condition de positivité. $$4x - 6 \geq 0$$

            $$\implies \mathbb{D}_{\text{pos.}} = \left[\frac{3}{2}; + \infty \right[$$

        \item Sur $\mathbb{D}_{\text{pos}}$, on peut élever au carré (équivalence!)

            $$x^{2} - 3x + 6 = (4x - 6)^{2} = 16 x^{2} - 48 x + 36$$

            \[
                \begin{split}
                    15 x ^{2} - 45 x + 30 &= 0\\
                    x^{2} - 3 x + 3 &= 0\\
                    (x -2 )\cdot (x -1 ) &= 0
                \end{split}
            \]
            $$x = 1 \text{ ou } x = 2$$

        \item Solution $$S = \mathbb{D}_{\text{déf}} \cap \mathbb{D}_{\text{pos.}}
            \cap \{1; 2\} = \{2\}$$
    \end{itemize}


    \item Résoudre en $x \in \mathbb{R}$ en fonction du paramètre $m \in \mathbb{R}$
    $$\sqrt{x + m^{2}} = x + m$$

    \begin{itemize}
        \item $\mathbb{D}_{\text{déf}} = [-m^{2}; + \infty[$
        \item Condition de positivité: $$x + m > 0$$
            $$\mathbb{D}_{\text{pos.}} = [-m; + \infty[$$

        \item Dans $\mathbb{D}_{\text{pos}}$ élever au carré:
            $$ x + m^{2} = (x + m)^{2} = x^{2} + 2xm + m^{2}$$
            \[
                \begin{split}
                    x^{2} + 2mx - x &=0\\
                    x\cdot (x +2m -1) &=0\\
                \end{split}
            \]

            $$x = 0 \text{ ou } x = 1-2m$$

        \item Voyons pour quelle valeur de $m$ les 2 valeurs sont solution
            \begin{itemize}
                \item $x = 0$ $$x = 0 \in \mathbb{D}_{\text{déf}} = [-m^{2}; +
                    \infty[$$
                    ceci est vérifié $\forall m \in \mathbb{R}$
                    $$x = 0 \in \mathbb{D}_{\text{pos.}} = [-m ; + \infty[$$
                    est vérifié $\forall m \geq 0$

                    Donc $x = 0$ est solution $\forall m \in \mathbb{R}_{+}$

                \item $x = 1 - 2m$ 
                    \[
                        \begin{split}
                            &1 - 2m \in [-m^{2}; + \infty[\\
                            \iff & 1 - 2 m \geq - m^{2}\\
                            \iff & m^{2} -2m + 1 \geq 0\\
                            \iff & (m-1) ^{2} \geq 0\\
                            \iff & m \in \mathbb{R}\\
                        \end{split}
                    \]

                    \[
                        \begin{split}
                            &1 - 2m \in [-m; + \infty[\\
                            \iff & 1 - 2 m \geq - m\\
                            \iff & m^{2} -2m + 1 \geq 0\\
                            \iff & m \leq 1\\
                        \end{split}
                    \]
                    Donc $x = 1 - 2m$ est solution si $m \leq 1$

                    \begin{center}
                        \begin{minipage}{.5\linewidth}
                            \resizebox{\textwidth}{!}{
% ask Zano for the slide
                            }
                        \end{minipage}
                        \begin{minipage}{.49\linewidth}
                            \setlength{\parskip}{.3em}
                            \begin{itemize}
                                \item $m < 0 : S = \{1-2m\}$
                                \item $m \in [\ 0, 1\ ] : S = \{0, 1-2m\}$
                                \item $m > 1 : S = \{0\}$
                            \end{itemize}
                        \end{minipage}
                    \end{center}
            \end{itemize}
        \end{itemize}

    \item Résoudre en $x \in \mathbb{R}$
        $$\sqrt{6 - x} \leq  3 + 2x$$

        \begin{itemize}
            \item $\mathbb{D}_{\text{déf}} = ]-\infty; 6\ ]$
            \item Condition de positivité $$3 + 2x \geq 0 \iff \mathbb{D}_{\text{pos.}}
                = \left[\frac{3}{2}; + \infty\right[$$
            \item Dans $\mathbb{D}_{\text{pos.}}$ élever au carré:
                $$6 -x \leq (3 + 2x)^{2} = 9 + 12 x + 4 x^{2}$$
                \[
                    \begin{split}
                        4x^{2} + 13 x + 3 & \geq 0\\
                        (4x + 1) \cdot (x +3) &\geq 0\\
                    \end{split}
                \]
                $$\implies S = ]-\infty; -3\ ] \cup \left[\ - \frac{1}{4};  + \infty\right[$$

            \item  Solution finale
                $$S_{\text{fin}} = \mathbb{D}_{\text{déf}} \cap
                \mathbb{D}_{\text{pos.}} \cap S = \left[-\frac{1}{4}; 6\right]$$
        \end{itemize}

    \item  Résoudre en $x \in \mathbb{R}$
        $$\sqrt{- x^{2} -x + 6} \geq x+1$$

        \begin{itemize}
            \item $\mathbb{D}_{\text{déf}}:$

                \[
                    \begin{split}
                        &-x^{2} - x + 6 \geq 0\\
                        \iff &? \\ %TODO second photo from zano
                        \iff &? \\
                    \end{split}
                \]

            \item Solution:

                \[
                    \begin{split}
                        S &= \mathbb{D}_{\text{déf}} \cap (S_{1} \cap S_{2})\\
                        &= [\ -3; 2 \ ] \cap ]-\infty; 1\ ] = [\ -3; 1\ ]
                    \end{split}
                \]
        \end{itemize}

\end{enumerate}


% 09/04/2019


\end{document}
