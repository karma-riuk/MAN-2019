%{{{-----------------------------------Basics----------------------------------%

%{{{document class definition
\documentclass[
    11pt,
    a4paper,
    oneside,
    headinlcude, footinclude,
    twoside,
]{report}
%}}}

%{{{essential packages
\renewcommand*\rmdefault{ppl}
\usepackage[top=2.5cm,bottom=2.5cm,left=3cm,right=3cm]{geometry}
\usepackage[english]{babel}
\usepackage[T1]{fontenc}
\usepackage[utf8]{inputenc}
\usepackage{xcolor}
\usepackage{amssymb}
%}}}

%{{{additional packages
\usepackage{amssymb}
\usepackage{amsmath}
\usepackage{mathtools}
\usepackage[framemethod=Tikz]{mdframed}
\usepackage{tikz}
\usepackage{enumerate}
\usepackage{graphicx}
\usepackage{pgf,tikz,pgfplots} % for the transfer from geogebra to tikz
\usepackage{mathrsfs}% for the transfer from geogebra to tikz
\usepackage{stackrel}
\usepackage{fancyhdr}
\usepackage{tabularx}
\usepackage{makecell} % to make thick hlines in the front page
\usepackage{centernot}
\usepackage[makeroom]{cancel}
%}}}

%}}}

%{{{-----------------------------------Macros----------------------------------%

\newcommand{\myImplies}[0]{\rightarrow}

\newcommand{\powerset}[1]{\mathcal{P}(#1)}

\newcommand{\tvect}[3]{%
   \ensuremath{\Bigl(\begin{smallmatrix}#1\#2\#3\end{smallmatrix}\Bigr)}}

\newcommand{\myVector}[3]{\begin{pmatrix}#1\#2\#3\end{pmatrix}}

\newcommand{\tq}[0]{\textrm{ t.q. }}

\newcommand{\markDate}[1]{\begin{flushright}#1\end{flushright}}

\newcommand{\cqfd}[0]{\begin{flushright}$\Box$\end{flushright}}

\renewcommand{\vec}[1]{\overrightarrow{#1}}

\def\getangle(#1)(#2)#3{
    \begingroup
        \pgftransformreset
        \pgfmathanglebetweenpoints{\pgfpointanchor{#1}{center}}{\pgfpointanchor{#2}{center}}
        \expandafter\xdef\csname angle#3\endcsname{\pgfmathresult}
    \endgroup
}

%colored frame box
\newcommand{\cfbox}[2]{
    \colorlet{currentcolor}{.}
    {\color{#1}
    \fbox{\color{currentcolor}#2}}
}

\newcommand\Warning{
    \makebox[1.4em][c]{
    \makebox[-5.5pt][c]{\raisebox{.2em}{!}}
    \makebox[0pt][c]{\color{red}\huge$\bigtriangleup$}}
}

%}}}

%{{{----------------------------------Settings---------------------------------%

\title{Maths 1B - Analyse}

\author{Arnò Fauconnet}

\setlength{\parindent}{0pt} %disable initial indent on first paragraph of sections in the whole doc

% the style of the boxes
\newmdenv[
        roundcorner=10pt,
        middlelinecolor=red,
        backgroundcolor=gray!15,
        linewidth=2pt,
        frametitlerule=true]{highlightBox}

% increases the space between paragraphs
\setlength{\parskip}{.3em}

%\setlength{\headheight}{15pt}

% geogebra library usage
\usetikzlibrary{arrows}
\pgfplotsset{compat=1.15}

% Geogebras wierd colors xD
\definecolor{qqqqff}{rgb}{0.,0.,1.}
\definecolor{ffqqtt}{rgb}{1.,0.,0.2}
\definecolor{ududff}{rgb}{0.30196078431372547,0.30196078431372547,1.} 
\definecolor{ffqqqq}{rgb}{1.,0.,0.} 
\definecolor{xdxdff}{rgb}{0.49019607843137253,0.49019607843137253,1.}
\definecolor{zzttqq}{rgb}{0.6,0.2,0.}
\definecolor{uuuuuu}{rgb}{0.26666666666666666,0.26666666666666666,0.26666666666666666}
\definecolor{qqzzqq}{rgb}{0.,0.6,0.}
\definecolor{ccqqqq}{rgb}{0.8,0.,0.}
\definecolor{qqwuqq}{rgb}{0.,0.39215686274509803,0.}
\definecolor{wwqqcc}{rgb}{0.4,0.,0.8}

\graphicspath{ {Maths_1B/figures/} }

%\tikzstyle{every node}[font=\large]


% header and footer settings
\pagestyle{fancy}
\fancyhf{}
\fancyhead[LE,LO]{Arnaud Fauconnet}
\fancyhead[CE,CO]{\textsc{Maths 1B}}
\fancyhead[RE,RO]{MAN - Printemps 2019}
\fancyfoot[CE,CO]{\leftmark}
\fancyfoot[LE,RO]{\thepage}

\renewcommand{\headrulewidth}{2pt}
\renewcommand{\footrulewidth}{1pt}

%}}}

%%%%%%%%%%%%%%%%%%%%%%%%%%%%%%%%%%%%%%%%%%%%%%%%%%%%%%%%%%%%%%%%%%%%%%%%%%%%%%
%----------------------------------------------------------------------------%
%-------------------------------Text starts here-----------------------------%
%----------------------------------------------------------------------------%
%%%%%%%%%%%%%%%%%%%%%%%%%%%%%%%%%%%%%%%%%%%%%%%%%%%%%%%%%%%%%%%%%%%%%%%%%%%%%%


\begin{document}

\begin{titlepage}
   \begin{center}
       \vspace*{\fill}

       {\Huge EPFL}\\ 
%----------------------------------------------------------------------------%
       \vfill
       {\huge MAN}\\ [1em]
       {\Large Mise à niveau}\\
%----------------------------------------------------------------------------%
        \vfill
        \begin{tabularx}{\textwidth}{X}
            \Xhline{3\arrayrulewidth}\\
        \end{tabularx}\\ [2em]
        {\Huge Maths 1B} \\ [1em]
        \textsc{\huge Prepa-033(b)} \\ [2em]
        \begin{tabularx}{\textwidth}{X}
            \Xhline{3\arrayrulewidth}\\
        \end{tabularx}\\ [2em]
%----------------------------------------------------------------------------%
        \vspace{.7cm}
        {\large
        \begin{tabularx}{.9\textwidth}{Xr}
            \textit{Student:} & \textit{Professor:}\\
            Arnaud \textsc{Fauconnet} & Olivier \textsc{Woringer}
        \end{tabularx}}
%----------------------------------------------------------------------------%
        \vfill
        {\Large Printemps - 2019}

%----------------------------------------------------------------------------%
        \vfill
        \includegraphics[width=7cm]{epfl-logo}

       \vfill
   \end{center} 
\end{titlepage} 
\setcounter{chapter}{0}
\chapter{Suite de nombres réels}

\section{Définitions}
\markDate{26/02/2019}

\paragraph{Définition}
\label{par:definition}

Une suite de nombres réels est une application de $\mathbb{N}$* dans $\mathbb{R}$

\begin{equation*}
\begin{split}
a : & \mathbb{N}^{\ast} \longrightarrow \mathbb{R}\\
& n \longmapsto a(n) = a_{n}
\end{split}
\end{equation*}

$a_{n}$ est le terme generale de la suite et $n$ est le rang de $a_{n}$.

La suite $a _{1}, a _{2}, a _{3}, ..., $ se note "$(a _{n})$"

\textbf{Exemples} 

\begin{enumerate}[1)]
\item La suite $(a _{n})$ définie par son terme général $a _{n} = \frac{1}{3n-7}$
est la suite $-\frac{1}{4}, -1, \frac{1}{2}, \frac{1}{5}, ...$

\item $..., -3, -2, -1, 0, 1, 2, 3, ...$ n'est pas une suite (pas de
premier élément). Mais $0, 1, -1, 2, -2, 3, -3, ...$ est une suite
dont l'ensemble des valeurs est $\mathbb{Z}$.

\item $a, a, a, a, ... \quad a \in \mathbb{R},\quad a _{n} = a \quad \forall n \in \mathbb{N} ^{\ast}$ 
est une suite constante.

\item Chercher le terme general des suites définie par les premiers
termes:
\begin{enumerate}[a)]
\item  $(a _{n}): \color{blue}\overbrace{\color{black}1, 6,}^\text{\color{blue}$5$}
\overbrace{\color{black}11, 16,}^\text{\color{blue}$5$}
\overbrace{\color{black}21, 26,}^\text{\color{blue}$5$}$

$a _{n} = 5n - 4$

\item  $b _{n}: \frac{2}{3}, \frac{5}{6}, \frac{10}{9},
\frac{17}{12}, \frac{26}{15}, \frac{37}{18}, ...$

Difference des nominateurs: $3, 5, 7, 9, 11, ...$\\
Difference des dénominateurs: $3, 3, 3, 3, 3, ...$

$b _{n} = \frac{n ^{2} + 1}{3n}$
\end{enumerate}

\item Suites définie par recurrence

$c _{n+1} = 2 - \frac{1}{c _{n}}, \quad c _{1} = \frac{3}{2}$

$c _{1} = \frac{3}{2}, \quad c _{2} = 2 - \frac{3}{2} = \frac{4}{3}, \quad c _{3} = 2
- \frac{3}{4} = \frac{5}{4}, \quad c _{4} = 2 - \frac{4}{5} = \frac{6}{5}$

Conjecture: $c _{n} = \frac{n+2}{n+1} \quad \forall n \in \mathbb{N} ^{\ast}$

Demonstration par récurrence:

\begin{itemize}
\item Verification: $c _{1} = \frac{3}{2} \quad \frac{n+2}{n+1}|_{n=1}
= \frac{3}{2} \quad \checkmark.$

\item Demonstration du pas de récurrence:
\begin{itemize}
\item \textbf{Hypothèse}:  $c _{n} = \frac{n+2}{n+1}$ pour
un $n \in \mathbb{N}$*

\item \textbf{Conclusion}: $c _{n+1} = \frac{n+3}{n+2}$

\item \textbf{Preuve}: 

$c _{n+1} = 2 - \frac{1}{c _{n}} = 2 - \frac{1}{\frac{n+2}{n+1}}
= 2 - \frac{n+1}{n+2} = \frac{2(n+1)-(n+1)}{n+2} =
\frac{n+3}{n+2}$
\end{itemize}
\cqfd
\end{itemize}
\end{enumerate}

\paragraph{Définitions}
\label{par:definitions}

\begin{enumerate}
\item $(a _{n})$ est majoré si

$$ \exists M \in \mathbb{R}  a_{n} \leq M, \quad \forall n \in
\mathbb{N}^{\ast} $$
(M : majorant)
\item $(a _{n})$ est minoré si 

$$\exists N \in \mathbb{R} \tq a _{n} \geq N, \quad \forall n \in
\mathbb{N} ^{\ast} $$

(N: minorant)

\item $(a_{n})$ est bornée si elle amdet un majorant \textbf{et} un
minorant

\item $(a_{n})$ est croissante si $a_{n+1} \geq a_{n} \quad \forall n \in \mathbb{N} ^{\ast}$

(Strictement croissante si $a_{n+1} \geq a_{n} \quad \forall n \in \mathbb{N} ^{\ast}$)

\item $(a_{n})$ est décroissant si $a_{n+1} \leq a_{n} \quad \forall n \in \mathbb{N} ^{\ast}$

(Strictement décroissante si $a_{n+1} \leq a_{n} \quad \forall n \in \mathbb{N} ^{\ast}$)

\item $(a_{n})$ est monotone si elle est croissant \textbf{ou} (ou exclusif) décroissant

(Strictement monotone si elle est croissant ou strictement décroissant)
\end{enumerate}

\paragraph{Exemples}
\label{par:exemples}

\begin{enumerate}
\item $a_{n} = \frac{1}{n}, \quad (a_{n})$ est strictement décroissante et
borné:

$$0 < a_{n} \leq 1$$

\item $b_{n} = \frac{n}{n+1} = \frac{n+1-1}{n+1} = 1 - \frac{1}{\underbrace{n+1}_\text{$>0$}}$

Donc $(b_{n})$ est strictement croissante.
\end{enumerate}

\section{Limite d'une suite}
\label{sec:limite_d_une_suite}

\paragraph{Définition}

On dit que la suite $(a_{n})$ converge vers $a \ (a \in \mathbb{R})$ si
$\forall \epsilon > 0$, il existe un seuil $$N \in \mathbb{N} ^{\ast} (N =
N(\epsilon)) \tq n > N \implies | a_{n} - a | < \epsilon$$ $$ | a_{n} - a | <
\epsilon \iff - \epsilon < a_{n} - a < \epsilon \iff a - \epsilon < a_{n} < a
+ \epsilon \iff a_{n} \in \ ]a-\epsilon, a+\epsilon[ $$

Cet intervalle est appelé $\epsilon$-voisinage de $a$. Si $(a_{n})$ admet
$(a_{n})$ est convergente sinon elle divergente.

\pagebreak

Définition plus intuitive de la limite de suite:

$ \lim\limits_{n \rightarrow \infty} a_{n} = a $ si et seulement si
$\epsilon$-voisinage de a contient persque tout les termes de la suite (tous
les termes sauf nombre fini)

\paragraph{Exemples}
\label{par:exemples}

\begin{enumerate}
\item $a_{n} = a \quad \forall n \in \mathbb{N} ^{\ast}$

$\lim\limits_{n \rightarrow \infty} a_{n} = a \quad [N \in \mathbb{N}
\textrm{* quelconque }\Box]$

\item Montrons que $\lim\limits_{n \rightarrow \infty} \frac{1}{n} = 0$

Soit $\epsilon > 0$, montrons que $\exists N \in \mathbb{N} ^{\ast} (N
= N(\epsilon)) \tq n \geq N \implies | \frac{1}{n} - 0 | < \epsilon$

$$| \frac{1}{n} - 0 | < \epsilon \iff | \frac{1}{n} | < \epsilon \iff
\frac{1}{n} < \epsilon \iff n > \frac{1}{\epsilon}$$

\item  La suite $(b_{n})$ définie par 
$$b_{n} = (-1)^{n} \frac{n}{n+1}, \quad n \in \mathbb{N} ^{\ast} $$

diverge car une infinitée de termes sont dans le voisinage de $(+1)$
ou de $(-1)$ selon que n est pair ou impair

\end{enumerate}

\textbf{Theorèmes importants (sans demonstration)}

\begin{enumerate}
\item Une suite qui converge admet une seule limite
\item Toute suite convergente est bornée.

La reciproque est fausse.\\ Contre-exemple: $a_{n} = (-1)^{n}$

\item Les règles de calcul:

Soit $(a_{n})$ et $(b_{n})$ convergentes,\\
$$ \lim_{n \to \infty} a_{n} = a, \lim_{n \to \infty} b_{n} = b $$
\begin{enumerate}
\item $ \lim\limits_{n \to \infty} |a_{n}| = |a|$

\item $ \lim\limits_{n \to \infty} (a_{n} + b_{n}) = a + b$

$ \lim\limits_{n \to \infty} (a_{n} - b_{n}) = a - b$

\item $ \lim\limits_{n \to \infty} (a_{n} \cdot b_{n}) = a \cdot b$

\item si $b_{n} \neq 0, \quad \forall n \in \mathbb{N}^{\ast}$, et
si $b \neq 0$,  alors:

$ \lim\limits_{n \to \infty} \frac{a_{n}}{b_{n}} = \frac{a}{b}$
\end{enumerate}

\markDate{01/03/2019}

\item \textbf{Théorème de comparaison}:
Soient $(a_{n})$ et $(b_{n})$ convergantes, $a_{n} \myImplies a, b_{n}
\myImplies b$

\begin{center}
Si $\exists n _{0} \in \mathbb{N}^{*} \tq a_{n} \leq b_{n}, \quad
\forall n \geq n_{0}$

alors $a \leq b$
\end{center}

\item \textbf{Théorème des deux gendarmes}:

Soient 3 suites $(g_{n})$, $(d_{n})$ et $(a_{n})$ telles que

$$\exists n_{0} \in \mathbb{N}^{*} \textrm{ avec } g_{n} \leq a_{n} \leq
d_{n}, \quad \forall n \geq n_{0}$$

Alors si

$$ \lim_{n \to \infty}  g_{n} = \lim_{n \to \infty} d_{n} = l$$

on a que $(a_{n})$ est convergente et 

$$ \lim_{n \to \infty} a_{n} = l$$

\item Corollaire des 2 gendarmes soit $(a_{n})$ tel que
        
$$\lim_{n \to \infty} | a_{n} | = 0  \implies \lim_{n \to \infty}
(a_{n}) = 0$$ 
$$\underbrace{- | a_{n} | }_{\to 0}\leq a_{n} \leq \underbrace{|
a_{n} |}_{\to 0} $$
\cqfd

\item Toute suite monotone est bornée et convergente.

Plus précisément:
\begin{itemize}
\item Toute suite croissante et majoré converge
\item Toute suite décroissante et minoré converge
\end{itemize}
\end{enumerate}

\paragraph{3 exemples importants}
\label{par:3_exemples_importants}

\begin{enumerate}
\item Soient $q \in \mathbb{R} \backslash \{\pm1\}, \quad(a_{n})$ la suite
définie par $a_{n} = q^{n}$

$$a_{n} = q^{n}: \Big\{\begin{smallmatrix} \textrm{diverge si } | q |
> 1\\ \textrm{converge vers } 0 \textrm{ si } | q | < 1
\end{smallmatrix}$$

Montrons que si $| q | < 1, \quad \lim\limits_{n \to \infty} q^{n} =
0$

$$ | a _{n} | = | q^{n} | = | q |^{n} \textrm{ or } | q | < 1$$ 

donc

$$\exists p > 0 \tq q = \frac{1}{1-p}$$

Donc

$$ | q | ^{n} = \frac{1}{(1+p)^{n}} \implies \frac{1}{| q |^{n}} =
(1+p)^{n}$$

$$ = 1 + n \cdot p + ... + p^{n} \geq n \cdot p, \quad \textrm{ donc }
| q | ^{n} \leq \frac{1}{n \cdot p} $$

$$| a _{n} | = | q | ^{n} \leq \frac{1}{n} \cdot \frac{1}{p}$$

$$ \underbrace{0}_{\to 0} \leq | a_{n} | \leq  \underbrace{\frac{1}{n}
\cdot \frac{1}{p}}_{\to 0}$$


D'après les 2 gendarmes: $\lim\limits_{n \to \infty} | a_{n} | = 0$

D'après son corollaire: $\lim\limits_{n \to \infty} a_{n} = 0$

\cqfd

\item La série géométrique

Soit $q \in \mathbb{R} \backslash \{\pm1\}$ et $a_{n}$ définie par 

$$a_{n} = 1 + q + q^{2} + ... + q^{n-1}$$

Définition par la récurrence

$$a_{n+1} = a_{n}+q^{n}, \quad a_{1} = 1, \quad n \in \mathbb{N}^{*}$$

On réécrit le terme $a_{n}$

\[
\begin{split}
a_{n} &= 1 + q + q^{2} + ... + q^{n-2} + q^{n-1}\\
q\cdot a_{n} &=  q + q^{2} + ... + q^{n-1} + q^{n}\\
a_{n} - q \cdot a_{n} &= 1 - q^{n}\\
a_{n} (1 - q) &= 1 - q^{n}
\implies a_{n} = \frac{1-q^{n}}{1-q}
\end{split}
\]

$(a_{n})$ converge si et seulement si $| q | < 1$ et dans ce cas $\lim\limits_{n
\to \infty} a _{n} = \frac{1}{1-q}$

Exemples:

\begin{enumerate}
\item 
\[
\begin{split}
a_{n} &= \frac{1}{2} + \frac{1}{4} + ... + \Big(\frac{1}{2}\Big)^{n} \\
a_{n} &= \frac{1}{2} \cdot \Big(1 + \frac{1}{2} + ... + \Big(\frac{1}{2}\Big)^{n-1}\Big)\\
a_{n} - \frac{1}{2} \cdot \frac{1-\frac{1}{2}}{1-\frac{1}{2}}
&= 1 - \Big(\frac{1}{2}\Big)^{n}\\
\textrm{et } \sum_{k = 1}^{\infty}
\Big(\frac{1}{2}\Big)^{k} &= \lim_{n \to \infty} a_{n}
= 1
\end{split}
\]
\item 
\[
\begin{split}
b_{n} &= \frac{1}{4} + \Big(\frac{1}{4}\Big)^{2} + ... + \Big(\frac{1}{4}\Big)^{n}\\
b_{n} &= \frac{1}{4} \cdot \frac{1-\Big(\frac{1}{4}\Big)^{n}}{1-\frac{1}{4}}
\implies \lim_{n \to \infty} b_{n} = \frac{1}{3}
\end{split}
\]
\end{enumerate}
\item Le nombre $e$

Soit  
\[
\begin{split}
e_{n} &= 1 + \frac{1}{1!} + \frac{1}{2!} + ... +
\frac{1}{n!}\\
e_{n} &= \sum^{\infty}_{k=1} \frac{1}{k!} \\
\frac{1}{k!} &= \frac{1}{1 \cdot 2 \cdot
\underbrace{3}_{>2} \cdot ... \cdot
\underbrace{k}_{>2}} \leq
\Big(\frac{1}{2}\Big)^{k-1}
\end{split}
\]

Donc

\[
\begin{split}
e_{n} &\leq 1 + \frac{1}{1} + \underbrace{\frac{1}{2}
+ \frac{1}{2^{2}} + ... + \frac{1}{2^{n-1}}}_{\leq
1}\\ 
e_{n} &\leq 3\\
\\
2 &\leq e_{n} \leq 3, \quad \forall n \geq 2
\end{split}
\]

Montrons encore que $(e_{n})$ est strictement croissante:

$$ e_{n+1} = e_{n} + \frac{1}{(n+1)!}, \quad \textrm{ donc } e_{n+1}
> e_{n}$$

$(e_{n})$ croissante et majoré: elle converge.

On note $e$ sa limite.

\[
\begin{split}
e_{n} &= 2.71828 \\
e_{n} &= \sum^{\infty}_{k=0} \frac{1}{k!}
\end{split}
\]

Autre caractéristique du nombre $e$:
                 
$$ e = \lim_{n \to \infty} \Big(1 + \frac{1}{n}\Big)^{n}$$
\end{enumerate}

\section{Limite infinie}
\label{sec:limite_infinie}

\paragraph{Définition}
La suite $(a_{n})$ tend vers $+ \infty$ lorsque $n \to \infty$ si:

$$\forall A > 0, \quad \exists N \in \mathbb{N}^{*} (N = N(A)) \tq n \geq N
\implies f(x) > A$$

On écrit alors 

$$ \lim_{n \to \infty} a_{n} = + \infty$$

On dit que $(a_{2})$ diverge vers $+ \infty$ 

De même 

$$ \lim_{n \to \infty} b_{n} = - \infty \iff \exists N \in \mathbb{N}^{*} (N =
N(B)) \tq n \geq N \implies a_{n} > B$$

\paragraph{Exemple}
\label{par:exemple}

Montrons que $(a_{n}) = (n^{2})$  diverge vers $+ \infty$.

Soit $A > 0$ donné, montrons qu'il existe $N \in \mathbb{N}^{*} \tq a_{n} > A
\textrm{ si } n \geq N $.

$$a_{n} > A \iff n ^{2} > A \iff n \in \ ]-\infty; \sqrt A [ \ \cup \ ] \sqrt
A; + \infty[ $$

Tout $N > \sqrt A$ convient car $n \leq N$ (avec $N \geq \sqrt A$) $\implies
n^{2} > A$


\paragraph{Théorème importants}
\label{par:theoreme_importants}

\begin{enumerate}
\item Si $a_{n} \xrightarrow[n \to \infty]{} + \infty$ et si $b_{n}$
converge ou $b_{n} \xrightarrow[n \to \infty]{} + \infty$, alors $(a_{n} +
b_{n}) \xrightarrow[n \to \infty]{} + \infty$

\item Si $a_{n} \xrightarrow[n \to \infty]{} + \infty$ et $\lambda \in \mathbb{R}$

\begin{itemize}
\item Si $\lambda > 0, (\lambda a_{n}) \xrightarrow[n \to \infty]{} + \infty$
\item Si $\lambda < 0, (\lambda a_{n}) \xrightarrow[n \to \infty]{} - \infty$
\end{itemize}

\item Si $a_{n} \xrightarrow[n \to \infty]{} + \infty$ et $b_{n}
\xrightarrow[n \to \infty]{} b > 0$ ou si $b_{n} \xrightarrow[n \to
\infty]{} + \infty$, alors $(a_{n}\cdot b_{n}) \xrightarrow[n \to
\infty]{} + \infty$

\item Si $a_{n} \xrightarrow[n \to \infty]{} + \infty$, alors $\frac{1}{a_{n}}
\xrightarrow[n \to \infty]{} 0$

\item Théorème du gendarme

Soit $(a_{n}), (b_{n}) \tq \exists n_{0} \in \mathbb{N}^{*}$ avec $a_{n}
\leq b_{n}, \quad \forall n \geq n_{0}$

si $\lim\limits_{n \to \infty} b_{n} = + \infty$, alors $\lim\limits_{n \to \infty} a_{n} = + \infty$

(de manière analogue pour $a_{n} \to - \infty$)
\end{enumerate}

\paragraph{Exemples}
\label{par:exemples}

Montrons que si $q > 1$, alors $(q^{n})$ diverge vers $+ \infty$

\[
\begin{split}
q > 1 \implies& \exists \ p > 0 \tq q = 1 + p\\
q^{n} = (1+p)^{n} =& 1 + n \cdot p + ... + p^{n} > n \cdot p\\
\textrm{Or }  \lim_{n \to \infty} np =&  p \cdot \lim_{n \to \infty} n
= p \cdot + \infty = + \infty (p > 0)
\end{split}
\]

Donc $(q^{n})$ diverge vers $+ \infty$

\paragraph{Cas d'indétermination}

\begin{itemize}
\item Si $a_{n} \xrightarrow[n \to \infty]{} + \infty$ et $b_{n}
\xrightarrow[n \to \infty]{} - \infty$

On ne peut rien dire à priori de 

$$ \lim_{n \to \infty} (a_{n} + b_{n})$$

\item Si  $a_{n} \xrightarrow[n \to \infty]{} + \infty$ et $b_{n}
\xrightarrow[n \to \infty]{} 0$

On ne peut rien dire à priori de 

$$ \lim_{n \to \infty} (a_{n} \cdot b_{n})$$
\end{itemize}


\end{document}
