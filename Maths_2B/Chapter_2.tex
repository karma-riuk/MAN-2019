%{{{-----------------------------------Basics----------------------------------%

%document class definition
\documentclass[
    11pt,
    a4paper,
    oneside,
    headinlcude, footinclude,
    twoside,
]{report}


%essential packages
\renewcommand*\rmdefault{ppl}
\usepackage[top=2.5cm,bottom=2.5cm,left=3cm,right=3cm]{geometry}
\usepackage[english]{babel}
\usepackage[T1]{fontenc}
\usepackage[utf8]{inputenc}
\usepackage{xcolor}
\usepackage{amssymb}


%additional packages
\usepackage{amssymb}
\usepackage{amsmath}
\usepackage[framemethod=Tikz]{mdframed}
\usepackage{fancyhdr}
\usepackage{graphicx}
\usepackage{tabularx}
\usepackage{makecell} % to make thick hlines in the front page
\usepackage[makeroom]{cancel}
%}}}

%{{{-----------------------------------Macros----------------------------------%

\newcommand{\myImplies}[0]{\rightarrow}
\newcommand{\powerset}[1]{\mathcal{P}(#1)}
\newcommand{\tvect}[3]}}

%{{{----------------------------------Settings---------------------------------%

\title{Maths 2B - Algèbre Linéaire}

\author{Arn\`o Fauconnet}

\setlength{\parindent}{0pt} %disable initial indent on first paragraph of sections in the whole doc

% the style of the boxes
\newmdenv[
        roundcorner=10pt,
        middlelinecolor=red,
        backgroundcolor=gray!15,
        linewidth=2pt,
        frametitlerule=true]{highlightBox}

% increases the space between paragraphs
\setlength{\parskip}{.3em}

\graphicspath{ {Maths_2B/figures/} }

% header and footer settings
\pagestyle{fancy}
\fancyhf{}
\fancyhead[LE,LO]{Arnaud Fauconnet}
\fancyhead[CE,CO]{\textsc{Maths 2B}}
\fancyhead[RE,RO]{MAN - Printemps 2019}
\fancyfoot[CE,CO]{\leftmark}
\fancyfoot[LE,RO]{\thepage}

\renewcommand{\headrulewidth}{2pt}
\renewcommand{\footrulewidth}{1pt}
%}}}

%%%%%%%%%%%%%%%%%%%%%%%%%%%%%%%%%%%%%%%%%%%%%%%%%%%%%%%%%%%%%%%%%%%%%%%%%%%%%%
%----------------------------------------------------------------------------%
%-------------------------------Text starts here-----------------------------%
%----------------------------------------------------------------------------%
%%%%%%%%%%%%%%%%%%%%%%%%%%%%%%%%%%%%%%%%%%%%%%%%%%%%%%%%%%%%%%%%%%%%%%%%%%%%%%


\begin{document}

\begin{titlepage}
   \begin{center}
       \vspace*{\fill}

       {\Huge EPFL}\\ 
%----------------------------------------------------------------------------%
       \vfill
       {\huge MAN}\\ [1em]
       {\Large Mise à niveau}\\
%----------------------------------------------------------------------------%
        \vfill
        \begin{tabularx}{\textwidth}{X}
            \Xhline{3\arrayrulewidth}\\
        \end{tabularx}\\ [2em]
        {\Huge Maths 2B} \\ [1em]
        \textsc{\huge Prepa-032(b)} \\ [2em]
        \begin{tabularx}{\textwidth}{X}
            \Xhline{3\arrayrulewidth}\\
        \end{tabularx}\\ [2em]
%----------------------------------------------------------------------------%
        \vspace{.7cm}
        {\large
        \begin{tabularx}{.9\textwidth}{Xr}
            \textit{Student:} & \textit{Professor:}\\
            Arnaud \textsc{Fauconnet} & Simon \textsc{Bossoney}
        \end{tabularx}}
%----------------------------------------------------------------------------%
        \vfill
        {\Large Printemps - 2019}

%----------------------------------------------------------------------------%
        \vfill
        \includegraphics[width=7cm]{epfl-logo}

       \vfill
   \end{center} 
\end{titlepage} 
\setcounter{chapter}{1}
\chapter{Check title of this chapter}
\label{cha:check_title_of_this_chapter}

\section{First sec }
\label{sec:first_sec}
\section{Second sec }
\label{sec:first_sec}

\section{Base et dimension}
Dans le cours précédent on a définit les notions de famille génératrice et de
famille libre d'un espace vectoriel donné. Une famille libre et génératrice
$\{v_{1}, v_{2}, ..., v_{n}\}$ permet tout vecteur de l'espace vectoriel comme
combinaison linéaire des vecteurs $\{v_{1}, v_{2}, ..., v_{n}\}$. Un vecteur
de $\text{Vect}(v_{1}, v_{2}, ..., v_{n})$ peut s'écrire que d'une manière
comme linéaire des vecteurs $\{v_{1}, v_{2}, ..., v_{n}\}$. Il est dès lors
naturel de chercher une famille de vecteurs qui sera simultanément libre et
génératrice.

\subsection{Définition}
\label{sub:definition}

Soit $V$ un espace vectoriel, on dit que $V$  est de \textbf{dimension finie}
si et seulement si il possède une famille de vecteurs $\{v_{1}, v_{2}, ...,
v_{n}\}$ finie et génératrice.

Pour une telle famille de vecteurs, on peut établir une première comparaison
entre famille libre et génératrice.

\paragraph{Lemme de Steinitz}
\label{par:lemme_de_steinitz}

Soit $V$ un espace vectoriel et $G = \{g_{1}, g_{2}, ..., g_{m}\}$ une famille
génératrice et $L = \{l_{1}, l_{2}, ..., l_{n}\}$ une famille libre. Alors $$n
\leq m$$
On peut maintenant  utiliser le lemme de Steinitz pour montrer que tout
espace vectoriel de dimension finie possède une famille libre et génératrice
(en admettant l'axiome du choix, on peut montrer plus généralement, que tout
espace vectoriel de dimension finie ou infinie possède une telle famille).

\subsection{Corollaire}
\label{sub:corollaire}

Soit $V$ un espace vectoriel non-trivial de dimension finie. Alors $V$ possède
une famille de vecteur libres et génératrice.


\paragraph{Démonstration}
\label{par:demonstration}

Si $V \neq \{0_{v}\}$, alors $V$ possède au moins un vecteur non-nul de dimension
finie, il existe alors une famille génératrice $G = \{g_{1}, g_{2}, ..., g_{n}\}$
avec $m \geq 1$

\begin{itemize}
    \item La famille $L_{1} := \{v_{1}\}$ est manifestement libre. Si elle est
        génératrice, le corollaire est démontré. Sinon il doit exister un
        vecteur non-nul $v_{2} \in V$ tel que $v_{2} \notin
        \text{Vect}(v_{2})$. En vertu du résultat antérieur, on a alors que $L_{2}
        := \{v_{1}, v_{2}\}$ est libre.
    \item On continue d'ajouter successivement des vecteurs non-générateurs
        par la famille libre, afin d'obtenir une chaine de famille libre.

        $$L_{1} \subset L_{2} \subset ... \subset L_{k} \subset ...$$
        où $L_{k}$ est libre et possède $k$ éléments. Cette cha\^ine ne peut
        pas être infinie, car par le lemme de Steinitz une famille libre ne
        peut pas avoir plus de $m$ éléments. Il existe donc une famille $L_{n}$
        maximale dans cette chaine. Celle-ci doit être génératrice, car sinon
        il existerait un vecteur non-nul $v \in \text{Vect}(v_{1}, v_{2}, ...,
        v_{n})$ et $L_{m} \cup \{v\}$ serait encore libre contre disant le
        fait que $L_{n}$ soit maximale.
        \cqfd
    \item Dans le cas de l'espace vecteur trivial $\{0_{v}\}$, on pose par
        convention par que $\text{Vect}(0_{v}) = \{0_{v}\}$ 
    \item L'ensemble vide $\emptyset$ est une famille libre, mais non espace
        vectoriel car il ne contient pas l'élément neutre.
\end{itemize}

\subsection{Définition}
\label{sub:definition}

Soit $V$ un espace vectoriel. Une famille $B \subset V $ qui est simultanément
libre et génératrice est appelée une base \textbf{base} de $V$.

\begin{itemize}
    \item Pour un vecteur $v \in V $ et une base $ B = \{v_{1}, v_{2}, ..., v_{n}\}$,
        il existe donc un seul n-uplet $(\lambda_{1}, ..., \lambda_{n})$ tels
        que 
        $$v = \lambda_{1} \cdot v_{1} + ... + \lambda_{n}\cdot v_{n} \quad \text{où}
        \quad \lambda_{i} \in \mathbb{R} \forall i = 1, ..., n$$

        Les nombres $\lambda_{1}, ..., \lambda_{n}$ sont appelé les \textbf{composantes} du
        vecteurs $v$ relativement à la base $B$ $$[v]_{B} = \myVector
        {\lambda_{1}}{\vdots}{\lambda_{n}}$$
\end{itemize}

\paragraph{Exemple:}

On considère les quatre polynômes:

\begin{center}
    \begin{tabular}{ll}
        $\bullet \quad p_{1}(x) = 1 + x$ & $\bullet \quad p_{2}(x) = x + x^{2}$\\
        $\bullet \quad p_{3}(x) = 1 - x^{2}$ & $\bullet \quad p_{1}(x) = 1 + x + x^{2}$\\
    \end{tabular}
\end{center}

\begin{itemize}
    \item Par construction la famille polynôme $\{p_{1}(x), p_{2}(x), p_{3}(x),
        p_{4}(x)\}$ est génératrice de $V$. On va chercher une base de
        l'espace vectoriel de $V$.
    \item Clairement, $\{p_{1}(x)\}$ est libre, mais $\text{Vect}(p_{1}(x))
        \neq V$. De plus, $p_{2}(x)$ n'est pas linéairement dépendant de $p_{1}(x)$.
        Ainsi, la famille $\{p_{1}(x), p_{2}(x)\}$ est libre.
    \item Par contre, $p_{3}(x) = p_{1}(x) - p_{2}(x)$. Ainsi, la famille
        $\{p_{1}(x), p_{2}(x), p_{3}(x)\}$ n'est pas une famille libre.
    \item On examine s'il y a une relation de dépendance linéaire entre les
        polynômes $p_{1}(x), p_{2}(x)$ et $p_{4}(x)$. Soit $p_{4}(x) = \alpha
        \cdot p_{1}(x) + \beta \cdot p_{2}(x)$
        $$1 + x + x^{2} = \alpha \cdot (1 + x) + \beta \cdot (x + x^{2}) =
        \alpha + (\alpha + \beta)\cdot x + \beta \cdot x^{2}$$
        Ainsi, 
        $$ \left\{
            \begin{array}{l}
            \alpha = 1\\
            \alpha + \beta = 1\\
            \beta = 1\\
            \end{array}
        \right. \implies  \text{ce système n'as pas de solutions. Donc, }
        p_{4}(x) \notin \text{Vect}(p_{1}(x), p_{2}(x))$$
    \item Par conséquent, la famille $\{p_{1}(x), p_{2}(x), p_{4}(x)\}$ est
        libre et génératrice tout $V$. En effet $$V = \text{Vect}(p_{1}(x),
        p_{2}(x),\cancel{p_{3}(x)}, p_{4}(x)) = \text{Vect}(p_{1}(x), p_{2}(x),
        p_{4}(x))$$

    \item On vient donc de trouver un base pour $V$.

    \item On peut aussi choisir de suivre le chemin inverse en commencant par
        la famille génératrice et en supprimant les éléments linéairement
        dépendant des autres.

    \item Les vecteurs de dépendance linéaire s'écrivent, 
        $$<\alpha \cdot p_{1}(x) + \beta \cdot p_{2}(x) + \gamma \cdot p_{3}(x)
        + \delta \cdot p_{4}(x) = 0>$$
        $$<(\alpha + \gamma + \delta) + (\alpha + \beta + \delta) \cdot x +
        (\beta - \gamma + \delta) \cdot x^{2} = 0>$$
    \item On est donc réduit à résoudre le système, 

        $$
        \left\{
            \begin{array}{l}
            \alpha + \gamma + \delta = 0\\
            \alpha + \beta + \delta = 0\\
            \beta - \gamma + \delta = 0\\
            \end{array}
        \right.
        \sim  
        \left(
            \begin{array}{cccc|c}
                {\color{red} \alpha} & {\color{red} \beta} & {\color{red} \gamma}
                &{\color{red} \delta}\\
                1 & 0 & 1 & 1 & 0\\
                1 & 1 & 0 & 1 & 0\\
                0 & 1 & -1 & 1 & 0\\
            \end{array}
        \right)
        $$
    \item Solution:
        $$\left( 
            \begin{array}{c} 
                \alpha\\
                \beta\\
                \gamma\\
                \delta\\ 
            \end{array} 
        \right) = 
        \left( 
            \begin{array}{c} 
                \lambda\\
                - \lambda\\
                -\lambda\\
                0\\ 
            \end{array} 
        \right) \text{ où } \lambda \in \mathbb{R}, \quad (\alpha = \lambda)
        $$

    \item La relation de dépendance linéaire devient 
        $$< \lambda \cdot p_{1} (x) - \lambda \cdot p_{2}(x) - \lambda \cdot
        p_{3}(x) = 0>$$

    \item Ainsi, on peut choisir d'éliminer un des trois polynôme $p_{1}(x),
        p_{2}(x)$ ou $p_{3}(x)$ (ce qui revient à poser que $\alpha = 0, \beta
        - 0$ ou $\gamma = 0$). Une fois que ce polynôme est éliminé, les trois
        polynôme restant seront linéairement indépendants. On a donc trouvé
        trois bases pour $V$:
        $$ \{p_{1}(x), p_{2}(x), p_{3}(x), p_{4}(x)\} \setminus \{p_{k}(x)\},
        \quad \text{ où } k = 1, 2, 3$$

    \item Une base permet d'identifier un espace vectoriel de dimension finie
        avec $\mathbb{R}^{n}$ (bijective).

    \item Deux bases différentes donneront des identifications différentes
        (composantes différentes).

    \item En particulier, un vecteur $v$ sera représenté par deux $n$-uplets
        différents dans deux bases différents (composantes différentes)

    \item Il ne faut pas confondre le vecteur $v$ avec le $n$-uplets qui le
        représente: \\le vecteur possède une existence en soi, alors que le $n$-uplets
        n'est est qu'un représentation qui dépend de la base choisie.

    \item Par contre, le nombre de $n$-uplets nécessaire pour identifier un
        vecteur donné est toujours le même. Il ne dépend pas de la base.
\end{itemize}

\subsection{Théorème}
\label{sub:theoreme}

Soient $V$ un espace vectoriel et $B_{1} = \{b_{1}, ..., b_{n}\}$ et
$B_{2} = \{c_{1}, ..., c_{m}\}$ deux bases libres pour $V$. Alors,
$$n = m$$

\paragraph{Démonstration:}

Puisque $B_{1}$ est une base, c'est en particulier une famille libre. Puisque
$B_{2}$ est une base, c'est en particulier une famille génératrice. Par le
lemme de Steinitz, on conclut alors que $$n \leq m$$
En renversant les rôles de $B_{1}$ et $B_{2}$, on montre alors que $$m \leq n$$
Ainsi, $$m = n$$

\subsection{Définition}
\label{sub:definition}

Soit $V$ un espace vectoriel de dimension finie. La dimension de $V$ est le
nombre d'éléments d'une base de $V$.

\paragraph{Exemples:}

\begin{enumerate}
    \item La famille $\{v_{1}, v_{2}, v_{3}, v_{4}, v_{5}\}$ est génératrice
        de $V$

    \item La famille $\{v_{2}, v_{3}, v_{4}\}$ est libre

    \item $< 2 v_{1} + v_{2} - v_{3} + v_{4} + v_{5} = 0>$

    \item $< 3 v_{1} -3 v_{2} + v_{3} - 4 v_{4} + 2v_{5} = 0>$
\end{enumerate}

On cherche a montrer que la famille $\{v_{2}, v_{3}, v_{4}\}$ est une base de
$V$ et on veut déterminer sa dimension.

\begin{itemize}
    \item Comme la famille $\{v_{2}, v_{3}, v_{4}\}$, il suffit de montrer
        qu'elle est génératrice de $V$.

    \item La famille $\{v_{1}, v_{2}, v_{3}, v_{4}, v_{5}\}$ est génératrice,
        ainsi $$V = \text{Vect}(v_{1}, v_{2}, v_{3}, v_{4}, v_{5}) $$

    \item D'après $3.:$ $$v_{5} = -2 v_{1} - v_{2} + v_{3} - v_{4}$$ Donc $$V =
        \text{Vect}(v_{1}, v_{2}, v_{3}, v_{4}, \cancel{v_{5}}) = \text{Vect}(v_{1}, v_{2}, v_{3}, v_{4})$$

    \item D'après $4.$: $$3 v_{1} - 3v_{2} + v_{3} - 4 v_{4} + 2 \cdot
        (\overbrace{-2v_{1} - v_{2} +v_{3} - v_{4}}^{=v_{5}}) = 0$$
        $$-v_{1} - 5 v_{2} + 3 v_{3} - 6 v_{4} = 0 \implies v_{1} = -5v_{1} +
        3v_{2} - 6v_{4}$$
        Donc $$V = \text{Vect}(\cancel{v_{1}}, v_{2}, v_{3}, v_{4}) =
        \text{Vect}(v_{2}, v_{3}, v_{4})$$

    \item Ainsi, la famille $\{v_{2}, v_{3}, v_{4}\}$ est génératrice donc
        base de $V$. Elle contient 3 éléments. La dimension de $V$ est de 3.

    \item On cherche à présent les coordonnées du vecteur $v_{5}$ dans la base
        $B = \{v_{2}, v_{3}, v_{4}\}$ $$v_{5} = -2v_{1} - v_{2} +v_{3} - v_{4}
        = -2 \cdot (-5v_{1} + 3v_{2} - 6v_{4}) - v_{2} + v_{3} - v_{4} = 9
        v_{2} - 5 v_{3} + 11 v_{4}$$

    \item Coordonnées des vecteurs $v_{2}, v_{3}, v_{4}$ et $v_{5}$ dans la
        base $B$ $$
            [v_{2}]_{B} = \left( 
                \begin{array}{c} 
                    1\\
                    0\\
                    0\\ 
                \end{array} 
            \right) \quad 
            [v_{3}]_{B} = \left( 
                \begin{array}{c} 
                    0\\
                    1\\
                    0\\ 
                \end{array} 
            \right) \quad 
            [v_{4}]_{B} = \left( 
                \begin{array}{c} 
                    0\\
                    0\\
                    1\\ 
                \end{array} 
            \right) 
            $$

            $$
                [v_{4}]_{B} =
                9 \cdot \left( 
                    \begin{array}{c} 
                        1\\
                        0\\
                        0\\ 
                    \end{array} 
                \right)
                -5 \cdot \left( 
                    \begin{array}{c} 
                        0\\
                        1\\
                        0\\ 
                    \end{array}
                \right)
                +11 \cdot \left( 
                    \begin{array}{c} 
                        0\\
                        0\\
                        1\\ 
                    \end{array}
                \right)
                = \left( 
                    \begin{array}{c} 
                        9\\
                        -5\\
                        11\\ 
                    \end{array} 
                \right)
            $$
\end{itemize}

\end{document}
