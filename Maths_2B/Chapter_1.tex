%{{{-----------------------------------Basics----------------------------------%

%document class definition
\documentclass[
    11pt,
    a4paper,
    oneside,
    headinlcude, footinclude,
    twoside,
]{report}


%essential packages
\renewcommand*\rmdefault{ppl}
\usepackage[top=2.5cm,bottom=2.5cm,left=3cm,right=3cm]{geometry}
\usepackage[english]{babel}
\usepackage[T1]{fontenc}
\usepackage[utf8]{inputenc}
\usepackage{xcolor}
\usepackage{amssymb}


%additional packages
\usepackage{amssymb}
\usepackage{amsmath}
\usepackage[framemethod=Tikz]{mdframed}
\usepackage{fancyhdr}
\usepackage{graphicx}
\usepackage{tabularx}
\usepackage{makecell} % to make thick hlines in the front page
\usepackage[makeroom]{cancel}
%}}}

%{{{-----------------------------------Macros----------------------------------%

\newcommand{\myImplies}[0]{\rightarrow}
\newcommand{\powerset}[1]{\mathcal{P}(#1)}
\newcommand{\tvect}[3]}}

%{{{----------------------------------Settings---------------------------------%

\title{Maths 2B - Algèbre Linéaire}

\author{Arn\`o Fauconnet}

\setlength{\parindent}{0pt} %disable initial indent on first paragraph of sections in the whole doc

% the style of the boxes
\newmdenv[
        roundcorner=10pt,
        middlelinecolor=red,
        backgroundcolor=gray!15,
        linewidth=2pt,
        frametitlerule=true]{highlightBox}

% increases the space between paragraphs
\setlength{\parskip}{.3em}

\graphicspath{ {Maths_2B/figures/} }

% header and footer settings
\pagestyle{fancy}
\fancyhf{}
\fancyhead[LE,LO]{Arnaud Fauconnet}
\fancyhead[CE,CO]{\textsc{Maths 2B}}
\fancyhead[RE,RO]{MAN - Printemps 2019}
\fancyfoot[CE,CO]{\leftmark}
\fancyfoot[LE,RO]{\thepage}

\renewcommand{\headrulewidth}{2pt}
\renewcommand{\footrulewidth}{1pt}
%}}}

%%%%%%%%%%%%%%%%%%%%%%%%%%%%%%%%%%%%%%%%%%%%%%%%%%%%%%%%%%%%%%%%%%%%%%%%%%%%%%
%----------------------------------------------------------------------------%
%-------------------------------Text starts here-----------------------------%
%----------------------------------------------------------------------------%
%%%%%%%%%%%%%%%%%%%%%%%%%%%%%%%%%%%%%%%%%%%%%%%%%%%%%%%%%%%%%%%%%%%%%%%%%%%%%%


\begin{document}

\begin{titlepage}
   \begin{center}
       \vspace*{\fill}

       {\Huge EPFL}\\ 
%----------------------------------------------------------------------------%
       \vfill
       {\huge MAN}\\ [1em]
       {\Large Mise à niveau}\\
%----------------------------------------------------------------------------%
        \vfill
        \begin{tabularx}{\textwidth}{X}
            \Xhline{3\arrayrulewidth}\\
        \end{tabularx}\\ [2em]
        {\Huge Maths 2B} \\ [1em]
        \textsc{\huge Prepa-032(b)} \\ [2em]
        \begin{tabularx}{\textwidth}{X}
            \Xhline{3\arrayrulewidth}\\
        \end{tabularx}\\ [2em]
%----------------------------------------------------------------------------%
        \vspace{.7cm}
        {\large
        \begin{tabularx}{.9\textwidth}{Xr}
            \textit{Student:} & \textit{Professor:}\\
            Arnaud \textsc{Fauconnet} & Simon \textsc{Bossoney}
        \end{tabularx}}
%----------------------------------------------------------------------------%
        \vfill
        {\Large Printemps - 2019}

%----------------------------------------------------------------------------%
        \vfill
        \includegraphics[width=7cm]{epfl-logo}

       \vfill
   \end{center} 
\end{titlepage} 
\setcounter{chapter}{0}
\chapter{Systèmes linéaires}
\label{cha:systemes_lineaires}

\section{Equations linéaires}
\label{sec:equations_lineaires}

\paragraph{Definition:} 
\label{par:definition}

Une \underline{équation linéaire}  à $n$ variables est une équation de type:

$$ a_{1} x_{1} + a_{2} x_{2} + ... + a_{n} x_{n} = b$$

où $a _{1}, ..., a _{n}, b \in \mathbb{R}$ sont \textbf{fixés} et où $x _{1},
..., x _{n} \in \mathbb{R}$ \textbf{variables}.
\\

Si \underline{$b = 0$}, on dit que l'équation est homogène.

Si \underline{$b \neq 0$}, on dit que l'équation est inhomogène.
\\

Une \textbf{solution} est donnée de $n$  nombres réels $x _{1}, ... ,x _{n}$
tel que l'équation est satisfaite.

La collection de $\mathcal{S}$ de toutes les solutions sera appelée
\textbf{l'ensemble des solutions} 

\paragraph{Exemple}

\begin{equation} \label{eq:1}
    x + 2y - 3z = 1 
\end{equation}

Équation linéaire à trois variables, $x, y, z, a _{1} = 1, a _{2} = 2, a _{3}
= -3, b = 1$

\paragraph{Resolution}

(\ref{eq:1}) est équivalente à 

$$ x = 1 - 2y + 3z$$

Si on rassemble $x, y \ \textrm{et}\ z$ en une colonne $\tvect{x}{y}{z}$, on a
que 

$$ \mathcal{S} = \Bigg\{ \myVector{x}{y}{z} \in \mathbb{R} ^{3} \ \Bigg|
\myVector{x}{y}{z} = \myVector{1}{0}{0} + y\myVector{-2}{1}{0} +
z\myVector{3}{0}{1}, y, z \in \mathbb{R} \Bigg\} $$

Choix $y = 1, z = 2$, 

$$ \myVector{x}{y}{z} = \myVector{1}{0}{0} + \myVector{-2}{1}{0} +
\myVector{6}{0}{1} = \myVector{5}{1}{2} $$

\paragraph{Remarque}
(\ref{eq:1}) est équivalente à $2y = 1 -x + 3z \ssi y = \frac{1}{2} - \frac{1}{2}x
+ \frac{3}{2}z$ 

\paragraph{On a alors:}
$$ \mathcal{S}' = \Bigg\{ \myVector{x}{y}{z} \in \mathbb{R} ^{3} \ \Bigg|
\myVector{x}{y}{z} = \myVector{0}{\frac{1}{2}}{0} + x\myVector{1}{-\frac{1}{2}}{0} +
z\myVector{0}{\frac{3}{2}}{1}, x, z \in \mathbb{R} \Bigg\} $$

$\mathcal{S} = \mathcal{S}'\ \textrm{si}\  x = 1, z = 2$, on a

$$ \myVector{x}{y}{z} = \myVector{1}{\frac{1}{2}}{0} + \myVector{1}{-\frac{1}{2}}{0} +
\myVector{0}{3}{2} = \myVector{1}{3}{2} $$

\markDate{4/03/2019}

\section{Système d'équation}
\label{sub:systeme_d_equation}

\paragraph{Definition}
\label{par:definition}

Un \textbf{système d'équation} linéaire est une famille de $m$ équation
linéaires à $n$ variables.\


\[
    \begin{split}
    &a_{11}\cdot x _{11} + ... + a_{1n} \cdot x_{1n} = b_{1}\\
    & \quad \vdots \\
    &a_{m1}\cdot x _{m1} + ... + a_{mn} \cdot x_{mn} = b_{m}\\
    \end{split}
\]

où $\{a_{ij}\}_{1 \leq i \leq m, 1 \leq j \leq m}$  et $\{b_{j}\}_{1 \leq j
\leq m}$ sont des \textbf{coefficients} (réels) du système.

\begin{center}
    $a_{i j}$

    $i$: ligne, $j$: colonne
\end{center}

Une solution est un $n$-tuple $\myVector{x_{1}}{\vdots}{x_{n}}$ qui satisfait
les $m$ équation simultanément.

Un système linéaires est complètement déterminé par ses coefficients.

Les coefficients seront dit "\textbf{triangulaire supérieur}" si 

$$a_{ij} = 0 \textrm{ dès que } i>j$$

\paragraph{Exemples}
\label{par:exemples}

\[
    \begin{split}
        {\color{blue} i = 1}&\quad x + y - z = 2\\
        {\color{blue} i = 2}&\quad {\color{red} 0x +} 2y + 3z = 1\\
        {\color{blue} i = 3}&\quad {\color{red} 0x + 0y +} z = 4\\
    \end{split}
\]

On "remonte" le système:

\[
    \begin{split}
        {\color{blue} i = 2}&\quad  2y + 12 = 1 \iff 2y = -11 \iff y = -\frac{11}{2}\\
        {\color{blue} i = 1}&\quad  x - \frac{11}{2} - 4 = 2 \iff x = \frac{23}{2}\\
        \\
        \implies S: \myVector{x}{y}{z} = \myVector{\frac{23}{2}}{\frac{11}{2}}{4}
    \end{split}
\]

On peut collecter les coefficients en un tableau:

\[
    \begin{split}
    5x - 2y + z &= 2\\
    3x + y + 2z - 4t &= 0\\
    2y + z + 2t &= 1
    \end{split}
\]

\begin{equation}
    \label{eq:2}
    \left(\begin{array}{cccc|c}
         5 & -2 & 1 & 0 & 2\\
         3 & 1 & 2 & -4 & 0\\
         0 & 2 & 1 & 2 & 1\\
    \end{array}\right)
\end{equation}

\begin{itemize}
    \item L'échange de deux lignes laisse l'ensemble des solutions $S$
        invariant.

    \item L'échange de deux colonnes (avec l'échange de variable
        correspondantes) laisse $S$ invariant.

    \item La multiplication d'une ligne entière par un nombre \textbf{non nul} 
        laisse $S$ invariant.

    \item L'addition de deux lignes laisse $S$ invariant.

    \item \textbf{À ne pas faire}:
        \begin{itemize}
            \item multiplier une colonne par un nombre
            \item additionner deux colonnes
        \end{itemize}
\end{itemize}


\[
    \begin{split}
        (\ref{eq:2}) \quad \sim \quad 
        \left(\begin{array}{cccc|c}
             1 & -\frac{2}{5} & \frac{1}{5} & 0 & \frac{2}{5}\\
             3 & 1 & 2 & -4 & 0\\
             0 & 2 & 1 & 2 & 1\\
        \end{array}\right) 
        &\quad L_{1} \leftarrow \frac{1}{5} L_{1}\\
        (\ref{eq:2}) \quad \sim \quad 
        \left(\begin{array}{cccc|c}
             1 & -\frac{2}{5} & \frac{1}{5} & 0 & \frac{2}{5}\\
             0 & \frac{11}{5} & \frac{7}{5} & -4 & -\frac{6}{5}\\
             0 & 2 & 1 & 2 & 1\\
        \end{array}\right) 
        &\quad L_{2} \leftarrow L_{2} - 3 L_{1}\\
        (\ref{eq:2}) \quad \sim \quad 
        \left(\begin{array}{cccc|c}
             1 & -\frac{2}{5} & \frac{1}{5} & 0 & \frac{2}{5}\\
             0 & \frac{11}{5} & \frac{7}{5} & -4 & -\frac{6}{5}\\
             0 & 2 & 1 & 2 & 1\\
        \end{array}\right) 
        &\quad L_{2} \leftarrow L_{2} - 3 L_{1}\\
        (\ref{eq:2}) \quad \sim \quad 
        \left(\begin{array}{cccc|c}
             1 & -\frac{2}{5} & \frac{1}{5} & 0 & \frac{2}{5}\\
             0 & 1 & \frac{7}{11} & - \frac{20}{11} & -\frac{6}{11}\\
             0 & 0 & 1 & - \frac{62}{3} & - \frac{23}{3}\\
         \end{array}\right) 
         &\quad \begin{array}{c} L_{2} \leftarrow \frac{5}{11} L_{2} \\ L_{3}
         \leftarrow L_{3} - 2 L_{2}\end{array}\\
    \end{split}
\]

\[
    \begin{split}
        z - \frac{62}{3}t = - \frac{23}{3} &\iff z = - \frac{23}{3} + \frac{62}{3}t\\
        y + \frac{7}{11} - \frac{20}{11}  t = - \frac{6}{11} &\iff y = -
        \frac{6}{11} - \frac{7}{11}  z + \frac{20}{11}  t\\
        y &= - \frac{6}{11} + \frac{161}{33} - \frac{441}{33}  t +
        \frac{20}{11}  t\\
        y &= \frac{143}{33} - \frac{374}{33}  t\\
        x - \frac{2}{5}y + \frac{1}{5}z = \frac{2}{5} &\iff x = \frac{2}{5} -
        \frac{2}{5} y + \frac{1}{5}z\\
        x = \frac{2}{5} + \frac{2}{5} \cdot \left(\frac{143}{33} - \frac{374}{33}t\right)&
        - \frac{1}{5} \cdot \left( - \frac{23}{3} + \frac{62}{3} t\right)\\
        x = \frac{2}{5} + \frac{286}{165} - &\frac{748}{165} t + \frac{25}{15}
        - \frac{62}{15}\\
        \\
        \left(\begin{array}{c}x \\y \\z \\t\end{array}\right) = 
        \left(\begin{array}{c}\frac{605}{165} \\\frac{143}{33} \\-\frac{22}{3} \\0\end{array}\right)
            &+t \cdot
        \left(\begin{array}{c}-\frac{1430}{165} \\-\frac{374}{33} \\-\frac{62}{3} \\1\end{array}\right)
    \end{split}
\]

\paragraph{Résumé}
\label{par:resume}

Supposons que le système d'équation est équivalent à un système triangulaire
supérieur à $k$ équation non nul à un $n$ variables.

\begin{itemize}
    \item L'ensemble des solutions est paramétré par $n-k$ paramètres (ex. $4-3=1$)
    \item Une solution a la forme:
        \[
            \left(\begin{array}{c}x_{1} \\\vdots \\x_{n}\end{array}\right) =
            \left(\begin{array}{c}c_{1} \\\vdots \\c_{n}\end{array}\right) + 
            \lambda_{1} \left(\begin{array}{c}v_{11} \\\vdots \\v_{1n}\end{array}\right) + \ldots + 
            \lambda_{n} \left(\begin{array}{c}v_{n n-k} \\\vdots \\v_{n n-k}\end{array}\right)
        \]
\end{itemize}

\paragraph{Exemple}
\label{par:exemple}

\[
    \left(\begin{array}{c}c_{1} \\c_{2} \\c_{3} \\c_{4}\end{array}\right) = 
    \left(\begin{array}{c}\frac{605}{165} \\\frac{143}{33} \\-\frac{22}{3} \\0\end{array}\right),
    \quad n - k = 1, \quad 
    \left(\begin{array}{c}v_{1} \\v_{2} \\v_{3} \\v_{4}\end{array}\right) = 
    \left(\begin{array}{c}-\frac{1430}{165} \\-\frac{374}{33} \\-\frac{62}{3} \\1\end{array}\right)
\]


\begin{itemize}
    \item Si $\left(\begin{array}{c}x_{1}\\ \vdots\\ x_{n}\end{array}\right)$
        et $\left(\begin{array}{c}y_{1}\\ \vdots\\ y_{n}\end{array}\right)$
        sont deux solutions alors
        $$t \cdot \left(\begin{array}{c}x_{1}\\ \vdots\\ x_{n}\end{array}\right)
         + (1-t) \cdot \left(\begin{array}{c}y_{1}\\ \vdots\\ y_{n}\end{array}\right)
            $$
        est encore une solution ($\forall t \in \mathbb{R}$)

        $S$ est, dans ce cas, dit "\textbf{convexe}"

    \item Si $\left(\begin{array}{c}x_{1}\\ \vdots\\ x_{n}\end{array}\right)$
        et $\left(\begin{array}{c}y_{1}\\ \vdots\\ y_{n}\end{array}\right)$
        sont deux solutions alors
        $$\left(\begin{array}{c}x_{1}\\ \vdots\\ x_{n}\end{array}\right)
         - \left(\begin{array}{c}y_{1}\\ \vdots\\ y_{n}\end{array}\right)$$
        est une solution à l'\textbf{équation homogène} (i.e. le système avec
        $b_{1} = \ldots = b_{n} = 0$)

    \item La partie paramètre de $S$ est l'ensemble des solutions un système
        homogène

    \item Cas où le système n'a pas de solutions ($S = \emptyset$)
        $$
            \left(\begin{array}{ccccc|c}
                 1 & x & x & \ldots & x &x\\
                 0 & 1 & x & x & x & x\\
                 \vdots & \ldots & 1 & x & x & x\\
                 {\color{red} 0} & {\color{red} 0} & {\color{red} 0} & {\color{red} 0} &{\color{red}  0} & {\color{red} x} \\
            \end{array}\right) 
        $$

    \item Considérons 

        $$
        \left\{\begin{array}{l}x - 3y + z = 1\\ x - 2y = 0 \\ x =1 \end{array} \right.
        \sim
        \left(\begin{array}{ccc|c}
             1 & -3 & 1 & 1\\
             1 & -2 & 0 & 0 \\
             1 & 0 & 0 & 1 \\
        \end{array}\right) 
        \sim
        \left( \begin{array}{ccc|c}
             1 & -3 & 1 & 1\\
             0 & -2 & 1 & 0 \\
             0 &  0 & 1 & 1 
        \end{array} \right) \quad  C_{1} \leftrightarrow C_{3}
        $$

\end{itemize}


 missing parts TODO: check the drive lesson 3 and 4 (normally)

\end{document}
